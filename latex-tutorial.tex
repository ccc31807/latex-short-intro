\documentclass[letterpaper]{article}

    \usepackage{framed}
    \usepackage{xcolor}
    \usepackage{listings}
    \usepackage{amsmath}
    \usepackage{amsfonts}
    \usepackage{mathtools}
    \usepackage{textcomp}
    \usepackage{lipsum}
    \usepackage[top=1in, bottom=1.5in, left=1.75in, right=1.75in]{geometry}
    \usepackage{hyperref}
    \usepackage{endnotes}

    \usepackage{imakeidx}
    \makeindex{}

    %this is a trick to print "\end{verbatim}" in a verbatim environment
    \usepackage{verbatim}
    \newenvironment{realverbatim}{\verbatim}{\endverbatim}

    %new commands
    \newcommand{\bib}{Bib\TeX}
    \newcommand{\Lx}{\LaTeXe}

    %\includeonly{math-symbols}

    \title{Another Short Tutorial for \LaTeXe}
    \author{Charles Carter}
    \date{\today{}}

\begin{document}

    \maketitle{}
    
    \tableofcontents{}
    \listoffigures{}
    \listoftables{}

    \newpage{}

    \section{Introduction}
    \label{Introduction}

    Let's begin thinking about \Lx{}  by considering the concept of \textit{easy}. We don't often think of easiness as having a context, but it certainly does. For example, which is easier to use, a shovel and wheelbarrow, or an excavator and dump truck? If we were digging a hole to plant a bush we would choose the former, but if we were digging a swimming pool we would choose the latter. Again, which is easier to use, a pickup truck or an eighteen wheeler? If we were hauling a few garden supplies we would choose the former, but if we were hauling electric generators to California, we would choose the latter. Which is easier to use, a word processing program (e.g., Microsoft Word), or a professional typesetting program (e.g., \Lx{})? If we were writing a letter or a grocery list, we would choose the former, but if we were writing a mathematical, scientific, or engineering document, we would choose the latter. I can't say whether one choice is ``easier'' than the other, but I can say that it's easier to use the right tool for the job than to attempt a task using the wrong tool. It's in this spirit that I offer this tutorial. 

    This is a short, easy, nontechnical introduction to \LaTeXe{}. It's a tutorial designed for students who do not have a lot of time, who do not need to become overnight experts in \LaTeXe{}, and prefer a shallow learning curve to a boot camp approach.  After you have worked through it, you will be able to create professional documents, and have the ability to teach yourself how to extend your \LaTeXe{} skills --- to become a \LaTeXe{} guru if you want to or need to.

    \textbf{What about \LaTeXe{}?} First, it's \textit{free} in the sense that you do not have to pay for it. Second, it's \textit{easy}, given that you know how to use it. As always, there are trade-offs, and ``easy'' to an expert tends to be hard for a beginner, and \textit{vice versa}. Third, it's \textit{stable}, the first version released in 1985. Fourth, it's \textit{well documented}, not surprising since its purpose is document preparation. Fifth, it's \textit{professioal}, as you will soon come to see. Finally, \Lx{} is \textit{universal}. It's pretty much everywhere, and can be reliably found on any computer that you use to do real work.I will touch on these points from time to time in this tutorial. 

    \textbf{How does this introduction work?} Each ``lesson'' consists of the introduction of a few commands, some text to copy, paste, and compile, and a couple of questions. Ech lesson should not take more than fifteen minutes to complete. It uses the ``baby talk'' principle --- you imitate what you see and explore it by making slight changes. If you complete one lesson a day, within several weeks you will have a good foundation with \LaTeXe{}, and begin to create professional quality documents. You will also see how the things I wrote in the preceding paragraph are true.

    Please note that this tutorial \textit{is not} a \LaTeXe{} reference; you will need to find a good reference that details the options and arguments for each command. It's also not a user guide. Nor is it a full introduction to \LaTeXe{}, although it will introduce it to students not previously familiar with it. It gives an ideosyncratic view of \LaTeXe{} (my own). It promises only to be short, easy, and useful, not long-winded or difficult, even if it does omit some necessary details. 

        \section{Document Basics}
    \label{Document-Basics}

    A \texttt{tex} document consists of plain text, and special characters, commands, and environments. I will generally refer to special characters, commands, and enviormments with the word \textit{commands}; you don't need to know the difference between them now, but you shortly will without being told. You \textit{must} precede commands with a backslash (\textbackslash{}) for the compiler to know that they are commands. This is easy to forget, so I will remind you the first couple of times.

	\paragraph{What should I have?}You should have a working \LaTeXe{} program. If you do not have one, see Appendix \ref{Installing} below. You will also need a text editor, and you will probably want to get an integrated editor, compiler, and printer. See Appendix \ref{Development Environments}. You can also use the old fashioned command line, I cover this in Appendix \ref{Command Line}.

        \subsection{Basic document}
        \label{Basic-document}
        
        \begin{framed}
            \begin{itemize}
                \item{documentclass}
                \item{begin/end document}
				\item{plain text}
				\item{comments}
            \end{itemize}
        \end{framed}

        A basic document begins with a document class, and has a preamble and contents. Type (or copy) the following, save it as a \texttt{.tex} document and compile it. The percent signs (\%) are comments and do not have any effect on  the document.

        \begin{verbatim}
\documentclass{article}
    %this is the preamble
\begin{document}
    %this is the contents section
    It works! %plain text prints as it
\end{document}    
        \end{verbatim}

        \paragraph{Exercise:}\LaTeXe{} has a number of different document classes. Name four of them.

        \paragraph{Exercise:}A \texttt{documentclass} command can take optional arguments, like this: \texttt{documentclass[optional arguments]\{document class\}}.\footnote{Don't forget to type a backslash before the command, like this: \texttt{\textbackslash{}documentclass}} Name two optional arguments.

        \subsection{Basic title}
        \label{Basic-title}
        
        \begin{framed}
            \begin{itemize}
                \item{title}
                \item{author}
                \item{date}
                \item{maketitle}
            \end{itemize}
        \end{framed}

        A basic document usually has a title and author information ib the preamble. Create and compile a second document like this:        

        \begin{verbatim}
\documentclass{article}
    \title{Title, Author, and Date}
    \author{Charles Carter}
    \date{July 4, 1776}
\begin{document} 
    \maketitle{}
    This document has a title, author, and date.
\end{document}    
        \end{verbatim}

        \paragraph{Exercise:}What happens if you use the command \texttt{today\{\}}\footnote{Remember, \texttt{\textbackslash{}date\{\}}} as the date parameter (replacing July 4, 1776)?

        \paragraph{Exercise:}What happens if you use the command \texttt{thanks\{email address\}}\footnote{\texttt{\textbackslash{}thanks\{\}}} after your name in the \texttt{author\{\}} command?

        \subsection{Basic sections}
        \label{Basic-sections}
        
        \begin{framed}
            \begin{itemize}
                \item{section}
                \item{subsection}
                \item{subsubsection}
                \item{label}
            \end{itemize}
        \end{framed}

        \LaTeXe{} provides a number of useful section levels, including part and chapter. Two of the most useful are \texttt{section} and \texttt{subsection}. Create and compile the following document. 

        The \texttt{label\{\}} is used to create cross-references in documents. It's also very helpful in organizing your thoughts. The argument to \texttt{label\{argument\}} does not appear in the document. I cover the \texttt{ref} and \texttt{pageref} commands below in subsection \ref{Footnotes} on page \pageref{Footnotes}. These are used to create references back to the label.

        \begin{verbatim}
\documentclass{article}
    \title{Basic Sections}
    \author{Charles Carter}
    \date{\today{}}
\begin{document} 
    \maketitle{}
    \section{Introduction}
    \label{Introduction}
    \section{Body}
    \label{Body}
    \section{Conclusion}
    \label{Conclusion}
\end{document}    
        \end{verbatim}

        \paragraph{Exercise:}What do the commands \texttt{subsection\{\}} and \texttt{subsubsection\{\}} do?

        \paragraph{Exercise:}What does \texttt{section*\{\}} do? Note the asterisk (*) after \texttt{section}. You can also use this starred version for \texttt{subsections} and \texttt{subsubsections}.

        \subsection{Basic paragraphs}
        \label{Basic-paragraphs}
        
        \begin{framed}
            \begin{itemize}
                \item{paragraph}
                \item{subparagraph}
            \end{itemize}
        \end{framed}


        We have reached the point where you need some real content. I will use the text of Abraham Lincoln's Gettysburg Address to illustrate paragraphs. Notice that ordinary paragraphs do not need a special commend -- the ``paragraph command'' is simply two blank lines to create an empty new line between the paragraphs, as if they were double spaced. Create and compile the following document.

        \begin{verbatim}
\documentclass{article}
    \title{Basic Paragraphs}
    \author{Charles Carter}
    \date{\today{}}
\begin{document} 
    \maketitle{}
    \section{Introduction}
    \label{Introduction}
    \section{Body}
    \label{Body}

Four score and seven years ago our fathers brought forth on this 
continent a new nation, conceived in liberty, and dedicated to 
the proposition that all men are created equal.

Now we are engaged in a great civil war, testing whether that nation, 
or any nation so conceived and so dedicated, can long endure. We 
are met on a great battlefield of that war. We have come to dedicate 
a portion of that field, as a final resting place for those who here 
gave their lives that that nation might live. It is altogether fitting 
and proper that we should do this.

But, in a larger sense, we can not dedicate, we can not consecrate, 
we can not hallow this ground. The brave men, living and dead, who 
struggled here, have consecrated it, far above our poor power to add 
or detract. The world will little note, nor long remember what we say 
here, but it can never forget what they did here. It is for us the 
living, rather, to be dedicated here to the unfinished work which they 
who fought here have thus far so nobly advanced. It is rather for us 
to be here dedicated to the great task remaining before us, that from 
these honored dead we take increased devotion to that cause for which 
they gave the last full measure of devotion, that we here highly resolve 
that these dead shall not have died in vain, that this nation, under 
God, shall have a new birth of freedom, and that government of the people, 
by the people, for the people, shall not perish from the earth.

    \section{Conclusion}
    \label{Conclusion}
\end{document}    
        \end{verbatim}
        
        \paragraph{Exercise:}What happens if you include the \texttt{paragraph\{\}} or \texttt{subparagraph\{\}} commands before each paragraph? 
        
        \paragraph{Exercise:}What happens if you include arguments with the \texttt{paragraph\{argument\}} or \texttt{subparagraph\{argument\}} commands? 

        \subsection{Basic packages}
        \label{Basic packages}
        
        \begin{framed}
            \begin{itemize}
                \item{usepackage}
                \item{lipsum}
            \end{itemize}
        \end{framed}

Much of \LaTeXe{} functionality is contained in external packages. To use this functionality, you include the command \texttt{usepackage\{\}} in the preamble. Of course, you first have to install the package on your computer, but the MiKTeX distribution does that automatically. The \texttt{lipsum} package generates generic text (in Latin, of course). The \texttt{lipsum\{\}} command generates text. Notice that you can control the number of paragraphs to include. Below, I hgave included paragraph $1$ in the introduction, paragraphs $2$ through $4$ in the body, and paragraph $5$ in the conclusion.

Notice the paragraph indentation. First paragraphs are \textit{not} indented. Following paragraphs \textit{are} indented. This is normal typographic practice.

        \begin{verbatim}
\documentclass{article}
    \usepackage{lipsum}
    \title{Using Packages}
    \author{Charles Carter}
    \date{\today{}}
\begin{document} 
    \maketitle{}
    \section{Introduction}
    \label{Introduction}
        \lipsum[1]{}
    \section{Body}
    \label{Body}
        \lipsum[2-4]{}
    \section{Conclusion}
    \label{Conclusion}
        \lipsum[5]{}
\end{document}    
        \end{verbatim}

        \paragraph{Exercise:}What is CTAN, the Comprehensive \TeX{} Archive Network? How many packages are currently on CTAN?

        \paragraph{Exercise:}What are the most popular \LaTeXe{} packages?

        \subsection{Basic contents}
        \label{Basic contents}
        
        \begin{framed}
            \begin{itemize}
                \item{tableofcontents}
            \end{itemize}
        \end{framed}

        Creating a table of contents is easy. Just include the \texttt{tableofcontents\{\}} command. You may have to compile the document twice to ensure that the table of contents is generated properly.
        \begin{verbatim}
\documentclass{article}
    \usepackage{lipsum}
    \title{Table of Contents}
    \author{Charles Carter}
    \date{\today{}}
\begin{document} 
    \maketitle{}
    \tableofcontents{}
    \section{Introduction}
    \label{Introduction}
        \lipsum[1]{}
    \section{Body}
    \label{Body}
        \lipsum[2-4]{}
    \section{Conclusion}
    \label{Conclusion}
        \lipsum[5]{}
\end{document}    
        \end{verbatim}

        \paragraph{Exercise:}The \texttt{section[argument]\{Section Title\}} command takes an optional argument. How does this argument affect the table of contents?

        \paragraph{Exercise:}What other kinds of content tables can \LaTeXe{} generate? To start with, look at figures and tables.

        \subsection{Basic decorations}
        \label{Basic decorations}
        
        \begin{framed}
            \begin{itemize}
                \item{textit}
                \item{textsf}
                \item{texttt}
                \item{textbf}
                \item{textsc}
                \item{underline}
            \end{itemize}
        \end{framed}

In this section, you will fiddle with the appearance of text. To \textit{create text in italics}, use \texttt{textit}. To \textsf{create text in sans serif}, use \texttt{textsf}. To \texttt{create text in monospace font}, use \texttt{texttt}. To \textbf{create text in boldface}, use \texttt{textbf}. To \textsc{create text using Small Caps}, use \texttt{textsc}. \underline{You should almost never underline text}! If you choose to do so, use \texttt{underline}.

        \paragraph{}To \textit{create text in italics}, use \texttt{textit}. 
        \paragraph{}To \textsf{create text in sans serif}, use \texttt{textsf}. 
        \paragraph{}To \texttt{create text in monospace font}, use \texttt{texttt}. 
        \paragraph{}To \textbf{create text in boldface}, use \texttt{textbf}. 
        \paragraph{}To \textsc{create text using Small Caps}, use \texttt{textsc}. 
        \paragraph{}\underline{You should almost never underline text}! If you choose to do so, use \texttt{underline}.

        \begin{verbatim}
\documentclass{article}
    \title{Font Appearance}
    \author{Charles Carter}
    \date{\today{}}
\begin{document} 
    \maketitle{}
    \section{Introduction}
    \label{Introduction}
    \section{Body}
    \label{Body}
        \paragraph{}In this section, you will fiddle with the appearance of text. 
        \paragraph{}To \textit{create text in italics}, use \texttt{textit}. 
        \paragraph{}To \textsf{create text in sans serif}, use \texttt{textsf}. 
        \paragraph{}To \texttt{create text in monospace font}, use \texttt{texttt}. 
        \paragraph{}To \textbf{create text in boldface}, use \texttt{textbf}. 
        \paragraph{}To \textsc{create text using Small Caps}, use \texttt{textsc}. 
        \paragraph{}\underline{You should almost never underline text}! 
        If you choose to do so, use \texttt{underline}.
    \section{Conclusion}
    \label{Conclusion}
\end{document}    
        \end{verbatim}

        \paragraph{Exercise:}As with much else in \LaTeXe,there are multiple ways to italisize or bold-face text. Can you find other ways?



        \subsection{Basic fontsizes}
        \label{Basic fontsizes}
        
        \begin{framed}
            \begin{itemize}
                \item{tiny}
                \item{scriptsize}
                \item{footnotesize}
                \item{small}
                \item{normalsize}
                \item{large}
                \item{Large}
                \item{LARGE}
                \item{huge}
                \item{Huge}
            \end{itemize}
        \end{framed}

\LaTeXe{} has several different ways to alter the size of the font. Perhaps the simplest way is to create a \textit{size environment}. You do this by using one of the commands listed above, and this controls the size of all text until it is changed by another command. You would typically use this for sections of text that need to be made smaller, such as tables, block quotes, technical sections not germane to the main discussion, and similar.

    \normalsize{}\paragraph{}This paragraph has a normalsize font size.
    \tiny{}\paragraph{}This paragraph has a tiny font size.
    \scriptsize{}\paragraph{}This paragraph has a scriptsize font size.
    \footnotesize{}\paragraph{}This paragraph has a footnotesize font size.
    \small{}\paragraph{}This paragraph has a small font size.
    \normalsize{}\paragraph{}This paragraph has a normalsize font size.
    \large{}\paragraph{}This paragraph has a large font size.
    \Large{}\paragraph{}This paragraph has a Large font size.
    \LARGE{}\paragraph{}This paragraph has a LARGE font size.
    \huge{}\paragraph{}This paragraph has a huge font size.
    \Huge{}\paragraph{}This paragraph has a Huge font size.
    \normalsize{}\paragraph{}This paragraph has a normalsize font size.

        \begin{verbatim}
\documentclass{article}
    \title{Font Sizes}
    \author{Charles Carter}
    \date{\today{}}
\begin{document} 
    \maketitle{}
    \section{Introduction}
    \label{Introduction}
    \section{Body}
    \label{Body}
    \normalsize{}\paragraph{}This paragraph has a normalsize font size.
    \tiny{}\paragraph{}This paragraph has a tiny font size.
    \scriptsize{}\paragraph{}This paragraph has a scriptsize font size.
    \footnotesize{}\paragraph{}This paragraph has a footnotesize font size.
    \small{}\paragraph{}This paragraph has a small font size.
    \normalsize{}\paragraph{}This paragraph has a normalsize font size.
    \large{}\paragraph{}This paragraph has a large font size.
    \Large{}\paragraph{}This paragraph has a Large font size.
    \LARGE{}\paragraph{}This paragraph has a LARGE font size.
    \huge{}\paragraph{}This paragraph has a huge font size.
    \Huge{}\paragraph{}This paragraph has a Huge font size.
    \normalsize{}\paragraph{}This paragraph has a normalsize font size.
    \section{Conclusion}
    \label{Conclusion}
\end{document}    
        \end{verbatim}

        \paragraph{Exercise:} The issues of font, font size, and font decoration, are difficult, complicated, and subject to internecene wars. You may want to postpone your exploration of these issues until you have created and compiled several hundred \texttt{.tex} documents. If you want, and have discretionary time available and nothing else to do, you may want to delve into the complex and divisive world of fonts, font sizes, and font decorations.



    \section*{\LaTeXe{} is \textit{Professional}}

    \LaTeXe{} is flexible to an extreme. You can customize every part of your document. However, its authors have spent much time and effort to ensure that it gives a professional appearance ``out of the box.'' This short tutorial does not cover customization. After you have created and compiled several dozen \texttt{.tex} files, you will begin to see the need for customization.\footnote{The first thing you may want to do is to change the page margins. An easy way to change page margins is to use the \texttt{geometry} package.} Unless you have special requirements, such as page margins, paragraph spacing, and page page numbers, and you are happy with the appearance of your document, you do not need to think about customization.

        \section{Math and Symbols}
    \label{Math}

	Both \TeX{} and \LaTeXe{} shine when it comes to math. In fact, Donald Knuth originally wrote \TeX{} just so he could typeset math. In this section, we will dip our toes into math and symbols. This will not be difficult. If you have need for more advanced mathematics, you will know how to find what you need to render your equations.

        \subsection{Special characters}
        \label{Special-characters}
        
		Most characters are not special. An \textit{a} is just an a, a \textit{Z} is just a Z, and a \textit{7} is just a 7. Sometimes, this isn't the case --- an \&  is not just an ampersand. \LaTeXe{} has ten special characters. They are listed below.

        \begin{framed}
			\textbackslash, \%, \{, \}, \$, \^{} , \_, \~{} , \#, \&
        \end{framed}

		You already lnow four of them. ``\textbackslash'' indicates the beginning of a command, ``\%'' indicates a comment, and the ``\{ - \}'' pair (usually) indicates the argument to a command. You will learn about three more in this section, ``\$'', ``\_'', and ``\^{}''. It's worthwhile to stare at these charcters long enough to become familiar with them. When you document misbehaves, often these characters are the culprit.

		Sometimes you will find characters that wish they were special, but are not. These include the cedilla (\c{c}), the degree (\textdegree), and diphthongs (\ae). All these are represented by \LaTeXe{} commands, you will use the command for the character.

		\paragraph{Exercise:}Scott Pakin has published the booklet \textit{The Comprehensiv \LaTeX{} Symbol List}. You can find this online in PDF format. Search for it and just look at it. It contains over 300 pages of symbols. You'll be amazed! 

        \subsection{Inline math}
        \label{Inline-math}
        
        \begin{framed}
            \begin{itemize}
                \item{\$}
                \item{plus or +}
				\item{- (dash or subtraction)}
                \item{times or ast}
                \item{frac or div}
                \item{sqrt}
				\item{\^{} (caret or circumflex)}
				\item{\_ (underscore)} 
            \end{itemize}
        \end{framed}

	This is an example of inline math. Use the dollar symbol (\$) to set the math. Here is how it works.  Addition: $4 + 5 = 9$.  Subtraction: $4 - 5 = -1$.  Multiplication: $4 \times 5 = 20$.  Multiplication: $4 \ast 5 = 20$.  Division: $\frac{4}{5} = 0.8$.  Division: $4 \div 5 = 0.8$.  Square root: $\sqrt{2} = 1.41421$.  Exponents: $2^8 = 256$.  Subscripts: $x_0, x_1, x_2$.

        \begin{verbatim}
\documentclass{article}
    \title{Inline Math}
    \author{Charles Carter}
    \date{\today{}}
\begin{document} 
    \maketitle{}
	This is an example of inline math. Use the dollar symbol (\$) to set the math. 
	Here is how it works. 
    Addition: $4 + 5 = 9$. 
    Subtraction: $4 - 5 = -1$. 
    Multiplication: $4 \times 5 = 20$. 
    Multiplication: $4 \ast 5 = 20$.
    Division: $\frac{4}{5} = 0.8$. 
    Division: $4 \div 5 = 0.8$. 
    Square root: $\sqrt{2} = 1.41421$. 
    Exponents: $2^8 = 256$.
	Subscripts: $x_0, x_1, x_2$
	\end{document}    
		\end{verbatim}

		\paragraph{Exercise:} You can find the \textit{User's Guide for the }\texttt{amsmath} \textit{Package} in PDF format online. Search for it and start reading through it.

        \subsection{Equations}
        \label{Equations}
        
        \begin{framed}
            \begin{itemize}
                \item{amsmath}
                \item{equation}
                \item{equation*}
            \end{itemize}
        \end{framed}
	
		\LaTeXe{} provides the \textit{equation} environment for writing block equations with the \textit{amsmath} package. First, import the package with \texttt{usepackage\{amsmath\}}.  Equations are numbered and can be referenced by means of their labels. The starred version omits the equation from the numbered equations. Here are some examples. Equation \ref{line} is the formula for a straight line. Equation \ref{slope} is the formula for the slope of a straight line.  The third, unnumbered equation is the formula for a straight line with multiple parameters.

		\begin{equation}
			\label{line}
			y = \beta_0 + \beta_1 x_1
		\end{equation}
		\begin{equation}
			\label{slope}
			m = \frac{y_1 - y_0}{x_1 - x_0}
		\end{equation}
		\begin{equation*}
			y = \beta_0 + \beta_1 x_1 + \beta_2 x_2 + \beta_3 x_3
		\end{equation*}

        \begin{verbatim}
\documentclass{article}
	\usepackage{amsmath}
    \title{Equations}
    \author{Charles Carter}
    \date{\today{}}
\begin{document} 
    \maketitle{}
	This is an example of equations.
		\begin{equation}
			\label{line}
			y = \beta_0 + \beta_1 x_1
		\end{equation}
		\begin{equation}
			\label{slope}
			m = \frac{y_1 - y_0}{x_1 - x_0}
		\end{equation}
		\begin{equation*}
			y = \beta_0 + \beta_1 x_1 + \beta_2 x_2 + \beta_3 x_3
		\end{equation*}
	\end{document}    
        \end{verbatim}

		\paragraph{Exercise:}Continue reading through the \textit{User's Guide for the }\texttt{amsmath} \textit{Package}.

        \subsection{Multiline equations}
        \label{Multiline equations}
        
        \begin{framed}
            \begin{itemize}
                \item{align}
                \item{gather}
                \item{multline}
            \end{itemize}
        \end{framed}

		How do I place several equations together in one equation environment, alighed on a particular character, such as an equal sign (=)? Use the \textit{align} environment, with the ampersand (\&) as the tab character, and end each line with two backslashs (\textbackslash{}\textbackslash{}).

		\begin{align}
			y& = \beta_0 + \beta_1 x_1\\
			slope& = \frac{y_1 - y_0}{x_1 - x_0}\\
			predictedvalue& = \beta_0 + \beta_1 x_1 + \beta_2 x_2 + \beta_3 x_3
		\end{align}

		How to I center the equations? Use the \textit{gather} environment, with no tab character but ending each line with two backslashes (\textbackslash{}\textbackslash{}).

		\begin{gather}
			y = \beta_0 + \beta_1 x_1\\
			slope = \frac{y_1 - y_0}{x_1 - x_0}\\
			predictedvalue = \beta_0 + \beta_1 x_1 + \beta_2 x_2 + \beta_3 x_3
		\end{gather}

		What if I have a very long equation that won't fit on one line? Use the \textit{multiine} environment, breaking with two backslashes (\textbackslash{}\textbackslash{})

		\begin{multline}
			y = \beta_0 + \beta_1 x_1 + \beta_2 x_2 + \beta_3 x_3 +\\
			\beta_4 x_4 + \beta_5 x_5 + \beta_6 x_6 +\\
			\beta_7 x_7 + \beta_8 x_8 + \beta_9 x_9
		\end{multline}
		
		\begin{verbatim}
\documentclass{article}
	\usepackage{amsmath}
    \title{Multiline Equations}
    \author{Charles Carter}
    \date{\today{}}
\begin{document} 
    \maketitle{}
	This is an example of align.
		\begin{align}
			y& = \beta_0 + \beta_1 x_1\\
			slope& = \frac{y_1 - y_0}{x_1 - x_0}\\
			predictedvalue& = \beta_0 + \beta_1 x_1 + \beta_2 x_2 + \beta_3 x_3
		\end{align}
	This is an example of gather.
		\begin{gather}
			y = \beta_0 + \beta_1 x_1\\
			slope = \frac{y_1 - y_0}{x_1 - x_0}\\
			predictedvalue = \beta_0 + \beta_1 x_1 + \beta_2 x_2 + \beta_3 x_3
		\end{gather}
	This is an example of multline.
		\begin{multline}
			y = \beta_0 + \beta_1 x_1 + \beta_2 x_2 + \beta_3 x_3 +\\
			\beta_4 x_4 + \beta_5 x_5 + \beta_6 x_6 +\\
			\beta_7 x_7 + \beta_8 x_8 + \beta_9 x_9
		\end{multline}
	\end{document}    
        \end{verbatim}

        \subsection{Higher math}
        \label{Higher math}
        
        \begin{framed}
            \begin{itemize}
                \item{sums}
                \item{products}
                \item{limits}
                \item{derivatives}
                \item{integrals}
            \end{itemize}
        \end{framed}

		Equation \ref{higher:sum} states the shorthand for the sum of a series of integers 1 through $n$. Equation \ref{higher:prod} states the shorthand for the product of a series of integeers 1 through $n$. Equation \ref{higher:lim} demonstrates the notation for limits. Equation \ref{higher:deriv} demonstrates the notation for derivatives. Equation \ref{higher:int} demonstrates the notation for integrals.

		\begin{gather}
			\sum_{i=1}^{i=n} i = 1 + 2 + 3 + \dotsm{} + n \label{higher:sum} \\
			\prod_{i=1}^{i=n} i = 1 \times 2 \times 3 \times \dotsm{} \times n \label{higher:prod} \\
			\lim_{x \to \infty} f(x) \label{higher:lim} \\
			\frac{d}{dx} \left[ e^\frac{x}{2} sin(ax)\right] \label{higher:deriv} \\
			\int_{a}^{b} x^2 dx \label{higher:int}
		\end{gather}
		

        \begin{verbatim}
\documentclass{article}
    \title{Higher Math}
    \author{Charles Carter}
    \date{\today{}}
\begin{document} 
    \maketitle{}
		\begin{gather}
			\sum_{i=1}^{i=n} i = 1 + 2 + 3 + \dotsm{} + n \\
			\prod_{i=1}^{i=n} i = 1 \times 2 \times 3 \times \dotsm{} \times n \\
			\lim_{x \to \infty} f(x) \label{higher:lim} \\
			\frac{d}{dx} \left[ e^\frac{x}{2} sin(ax)\right] \\
			\int_{a}^{b} x^2 dx
		\end{gather}
\end{document}    
        \end{verbatim}

		\paragraph{Exercise:}Continue reading through the \textit{User's Guide for the }\texttt{amsmath} \textit{Package}.

        \subsection{Matrices}
        \label{Matrices}
        
        \begin{framed}
            \begin{itemize}
                \item{vmatrix}
                \item{pmatrix}
                \item{bmatrix}
                \item{Bmatrix}
                \item{Vmatrix}
            \end{itemize}
        \end{framed}

        Matrices are contained within a math environment, either an equation block or inline math pairs (\$\ldots\$). They are entered by row. Each row ends with two backslash symbols (\textbackslash\textbackslash). Each element on a row is separaated by an ampersand (\&). The different variations create different surrounding brackets.
        \texttt{vmatrix} uses vertical bars ($|\ldots|$).
        \texttt{pmatrix} uses parentheses [(\ldots)].
        \texttt{bmatrix} uses square brackets ([\ldots]).
        \texttt{Bmatrix} uses curly braces (\{\ldots\}).
        \texttt{Vmatrix} uses double vertical bars ($||\ldots||$).

        $ \begin{vmatrix}
            1 & 3 & 3 \\
            1 & 4 & 3 \\
            1 & 3 & 4 
        \end{vmatrix}
         \times 
         \begin{vmatrix}
             7 & -3 & -3 \\
             -1 & 1 & 0 \\
             -1 & 0 & 1
         \end{vmatrix}
         =
         \begin{vmatrix}
             1 & 0 & 0 \\
             0 & 1 & 0 \\
             0 & 0 & 1
         \end{vmatrix} $

        \begin{verbatim}
\documentclass{article}
    \title{Matrices}
    \author{Charles Carter}
    \date{\today{}}
\begin{document} 
    \maketitle{}
        $ \begin{vmatrix}
            1 & 3 & 3 \\
            1 & 4 & 3 \\
            1 & 3 & 4 
        \end{vmatrix}
         \times 
         \begin{vmatrix}
             7 & -3 & -3 \\
             -1 & 1 & 0 \\
             -1 & 0 & 1
         \end{vmatrix}
         =
         \begin{vmatrix}
             1 & 0 & 0 \\
             0 & 1 & 0 \\
             0 & 0 & 1
         \end{vmatrix} $
\end{document}    
        \end{verbatim}

		\paragraph{Exercise:}Continue reading through the \textit{User's Guide for the }\texttt{amsmath} \textit{Package}.

        \subsection{Greek letters}
        \label{Greek letters}
        
%        \begin{framed}
%            \begin{itemize}%            \begin{itemize}
%                \item{}
%            \end{itemize}
%        \end{framed}

        Here are the Greek letters commonly used in mathematical and scientific applications. The \LaTeXe{} commands for the Greek letters are in inline math mode, so you \textit{must} surround the command by a pair of dollar signs (\$\ldots\$). You must also use the \texttt{amsmath} package.

        \begin{tabular}{|| l || l | l || l | l ||}
            \hline
            Alpha & Alpha & $A$      & alpha & $\alpha$ \\
            \hline
            Beta & B & $B$     & beta & $\beta$ \\
            \hline
            Gamma & Gamma & $\Gamma$     & gamma & $\gamma$ \\
            \hline
            Delta & Delta & $\Delta$     & delta & $\delta$ \\
            \hline
            Epsilon & E & $E$     & epsilon & $\epsilon$ \\
            \hline
            Zeta & Z & $Z$     & zeta & $\zeta$ \\
            \hline
            Eta & H  & $H$     & eta & $\eta$ \\
            \hline
            Theta & Theta & $\Theta$    & theta & $\theta$ \\
            \hline
            Iota & I & $I$     & iota & $\iota$ \\
            \hline
            Kappa & K & $K$     &  kappa & $\kappa$ \\
            \hline
            Lamba & Lambda & $\Lambda$    & lambda & $\lambda$ \\
            \hline
            Mu  & M & $M$     & mu & $\mu$ \\
            \hline
            Nu & N & $N$     & nu & $\nu$ \\
            \hline
            Xi & Xi & $\Xi$     & xi & $\xi$ \\
            \hline
            Omicron & O & $O$    & o & $o$ \\
            \hline
            Pi & Pi & $\Pi$     & pi & $\pi$ \\
            \hline
            Rho & R & $R$    & rho & $\rho$ \\
            \hline
            Sigma & S & $S$    & sigma & $\sigma$ \\
            \hline
            Tau & T & $T$    & tau & $\tau$ \\
            \hline
            Upsilon & Y & $Y$    & upsilon & $\upsilon$ \\
            \hline
            Phi & Phi & $\Phi$     & phi & $\phi$ \\
            \hline
            Chi & X & $X$     & chi & $\chi$ \\
            \hline
            Psi & Psi & $\Psi$     & psi & $\psi$ \\
            \hline
            Omega & Omega & $\Omega$    & omega & $\omega$ \\
            \hline
        \end{tabular}

        \begin{verbatim}
\documentclass{article}
    \usepackage{amsmath}
    \title{Greek Letters}
    \author{Charles Carter}
    \date{\today{}}
\begin{document} 
    \maketitle
        \begin{tabular}{|| l || l | l || l | l ||}
            \hline
            Alpha & Alpha & $A$      & alpha & $\alpha$ \\
            \hline
            Beta & B & $B$     & beta & $\beta$ \\
            \hline
            Gamma & Gamma & $\Gamma$     & gamma & $\gamma$ \\
            \hline
            Delta & Delta & $\Delta$     & delta & $\delta$ \\
            \hline
            Epsilon & E & $E$     & epsilon & $\epsilon$ \\
            \hline
            Zeta & Z & $Z$     & zeta & $\zeta$ \\
            \hline
            Eta & H  & $H$     & eta & $\eta$ \\
            \hline
            Theta & Theta & $\Theta$    & theta & $\theta$ \\
            \hline
            Iota & I & $I$     & iota & $k\iota$ \\
            \hline
            Kappa & K & $K$     &  kappa & $\kappa$ \\
            \hline
            Lamba & Lambda & $\Lambda$    & lambda & $\lambda$ \\
            \hline
            Mu  & M & $M$     & mu & $\mu$ \\
            \hline
            Nu & N & $N$     & nu & $\nu$ \\
            \hline
            Xi & Xi & $\Xi$     & xi & $\xi$ \\
            \hline
            Omicron & O & $O$    & o & $o$ \\
            \hline
            Pi & Pi & $\Pi$     & pi & $\pi$ \\
            \hline
            Rho & R & $R$    & rho & $\rho$ \\
            \hline
            Sigma & S & $S$    & sigma & $\sigma$ \\
            \hline
            Tau & T & $T$    & tau & $\tau$ \\
            \hline
            Upsilon & Y & $Y$    & upsilon & $\upsilon$ \\
            \hline
            Phi & Phi & $\Phi$     & phi & $\phi$ \\
            \hline
            Chi & X & $X$     & chi & $\chi$ \\
            \hline
            Psi & Psi & $\Psi$     & psi & $\psi$ \\
            \hline
            Omega & Omega & $\Omega$    & omega & $\omega$ \\
            \hline
        \end{tabular}
\end{document}    
        \end{verbatim}

        \paragraph{Exercise:} Mathematics uses a large number of non-Latin characters. For example, search for the Hebrew character Aleph ($\aleph$) and the symbol for infinity ($\infty$).



    \section*{\LaTeXe{} is \textit{Well Documented}}

    %references.tex
	
	\section{References}
	\label{References}

	\subsection{Footnotes}
	\label{Footnotes}


	\subsection{Endnotes}
	\label{Endnotes}


	\subsection{Margin Notes}
	\label{Margin Notes}


	\subsection{Bibliography}
	\label{Bibliography}


	\subsection{Citations}
	\label{Citations}


	\subsection{y}
	\label{y}

%        \subsection{}
%        \label{}
%        
%        \begin{framed}
%            \begin{itemize}
%                \item{}
%            \end{itemize}
%        \end{framed}
%
%
%        \begin{verbatim}
%\documentclass{article}
%    \title{This is My Title}
%    \author{Charles Carter}
%    \date{\today{}}
%\begin{document} 
%    \maketitle{}
%    \section{Introduction}
%    \label{Introduction}
%    \section{Body}
%    \label{Body}
%    \section{Conclusion}
%    \label{Conclusion}
%\end{document}    
%        \end{verbatim}
%
%        \paragraph{Exercise:}
%
%        \paragraph{Exercise:}




    \section*{\LaTeXe{} is \textit{Stable}}

    %lists.tex
	
	\section{Lists}
	\label{Lists}

        \subsection{Verbatim}
        \label{Verbatim}
        
        \begin{framed}
            \begin{itemize}
                \item{verbatim environment}
            \end{itemize}
        \end{framed}

    Perhaps the simplest and easiest way to make a list is by using the \texttt{verbatim} environment. Everything in the environment is printed as is, that is, \LaTeXe{} commands are not evaluated.

        \begin{realverbatim}
\documentclass{article}
    \title{Verbtim Environment}
    \author{Charles Carter}
    \date{\today{}}
\begin{document} 
    \maketitle{}
    \section{Verbatim Environment}
    \begin{verbatim}
        \texttt{one}
        \textit{two}
        \textsf{three}
        \huge{aye}
        \tiny{bee}
        \section{see}
    \end{verbatim}
\end{document}    
        \end{realverbatim}

        \paragraph{Exercise:} The only command not permitted in the \texttt{verbatim} environment is the \texttt{\textbackslash{}end\{verbatim\}} command. The reason is that this command ends the verbatim environment. If you examine the source of this tutorial, you will see that I resorted to a trick to include this command. As an exercise, can you figure out the trick?

        \subsection{Unordered Lists}
        \label{Unordered Lists}
        
        
%        \begin{framed}
%            \begin{itemize}
%                \item{}
%            \end{itemize}
%        \end{framed}
%
%
%        \begin{verbatim}
%\documentclass{article}
%    \title{This is My Title}
%    \author{Charles Carter}
%    \date{\today{}}
%\begin{document} 
%    \maketitle{}
%    \section{Introduction}
%    \label{Introduction}
%    \section{Body}
%    \label{Body}
%    \section{Conclusion}
%    \label{Conclusion}
%\end{document}    
%        \end{verbatim}
%
%        \paragraph{Exercise:}
%
%        \paragraph{Exercise:}


        \subsection{Ordered Lists}
        \label{Ordered Lists}
        
        
%        \begin{framed}
%            \begin{itemize}
%                \item{}
%            \end{itemize}
%        \end{framed}
%
%
%        \begin{verbatim}
%\documentclass{article}
%    \title{This is My Title}
%    \author{Charles Carter}
%    \date{\today{}}
%\begin{document} 
%    \maketitle{}
%    \section{Introduction}
%    \label{Introduction}
%    \section{Body}
%    \label{Body}
%    \section{Conclusion}
%    \label{Conclusion}
%\end{document}    
%        \end{verbatim}
%
%        \paragraph{Exercise:}
%
%        \paragraph{Exercise:}


        \subsection{Dictionary Lists}
        \label{Dictionary Lists}
        
%        \begin{framed}
%            \begin{itemize}
%                \item{}
%            \end{itemize}
%        \end{framed}
%
%
%        \begin{verbatim}
%\documentclass{article}
%    \title{This is My Title}
%    \author{Charles Carter}
%    \date{\today{}}
%\begin{document} 
%    \maketitle{}
%    \section{Introduction}
%    \label{Introduction}
%    \section{Body}
%    \label{Body}
%    \section{Conclusion}
%    \label{Conclusion}
%\end{document}    
%        \end{verbatim}
%
%        \paragraph{Exercise:}
%
%        \paragraph{Exercise:}

        \subsection{Listings Package}
        \label{Listings Package}
        
%        \begin{framed}
%            \begin{itemize}
%                \item{}
%            \end{itemize}
%        \end{framed}
%
%
%        \begin{verbatim}
%\documentclass{article}
%    \title{This is My Title}
%    \author{Charles Carter}
%    \date{\today{}}
%\begin{document} 
%    \maketitle{}
%    \section{Introduction}
%    \label{Introduction}
%    \section{Body}
%    \label{Body}
%    \section{Conclusion}
%    \label{Conclusion}
%\end{document}    
%        \end{verbatim}
%
%        \paragraph{Exercise:}
%
%        \paragraph{Exercise:}

        \subsection{Listings Version 1}
        \label{Listings Version 1}
        
%        \begin{framed}
%            \begin{itemize}
%                \item{}
%            \end{itemize}
%        \end{framed}
%
%
%        \begin{verbatim}
%\documentclass{article}
%    \title{This is My Title}
%    \author{Charles Carter}
%    \date{\today{}}
%\begin{document} 
%    \maketitle{}
%    \section{Introduction}
%    \label{Introduction}
%    \section{Body}
%    \label{Body}
%    \section{Conclusion}
%    \label{Conclusion}
%\end{document}    
%        \end{verbatim}
%
%        \paragraph{Exercise:}
%
%        \paragraph{Exercise:}

        \subsection{Listings Version 2}
        \label{Listings Version 2}
        
%        \begin{framed}
%            \begin{itemize}
%                \item{}
%            \end{itemize}
%        \end{framed}
%
%
%        \begin{verbatim}
%\documentclass{article}
%    \title{This is My Title}
%    \author{Charles Carter}
%    \date{\today{}}
%\begin{document} 
%    \maketitle{}
%    \section{Introduction}
%    \label{Introduction}
%    \section{Body}
%    \label{Body}
%    \section{Conclusion}
%    \label{Conclusion}
%\end{document}    
%        \end{verbatim}
%
%        \paragraph{Exercise:}
%
%        \paragraph{Exercise:}


    \section*{\LaTeXe{} is \textit{Easy}}

    %tables.tex
	
	\section{Tables}
	\label{Tables}

        Unfortunately, tables in \Lx{} tend to be a bit cumbersome. Tables are specified row by row, and require meticulous attention to detail. If sufficient care is taken, composing tables is not very difficult. Fortunately, simple tables are not too hard. 
        
        \subsection{Verbatim table}
        \label{Verbatim table}
        
        \begin{framed}
            \begin{itemize}
                \index{verbatim}
                \index{verbatim environment}
                \item{verbatim}
                \item{verbatim environment}
            \end{itemize}
        \end{framed}

        As with lists, ths simplest, most brain-dead way to make a table is by using the \texttt{verbatim} environment. The hardest part is making sure that your cells align properly using correct spacing. Hint: use a monospace font in your text editor rather than a variable spaced font.

        \begin{verbatim}
Abigail     Derek       Gladiola
Brenda      Edgar       Hibiscus
Claudia     Frank       Impatiens
        \end{verbatim}

        \begin{realverbatim}
\documentclass{article}
    \title{Verbatim Tables}
    \author{Charles Carter}
    \date{\today{}}
\begin{document} 
    \maketitle{}
        \begin{verbatim}
Abigail     Derek       Gladiola
Brenda      Edgar       Hibiscus
Claudia     Frank       Impatiens
        \end{verbatim}
\end{document}    
        \end{realverbatim}

        \paragraph{Exercise:} Create your own verbatim table.

        \subsection{Simple table}
        \label{Simple table}
        
        \begin{framed}
            \begin{itemize}
                \index{tabular environment}
                \index{l (column specifier)}
                \index{c (column specifier)}
                \index{r (column specifier)}
                \index{\textbackslash\textbackslash (row terminator)}
                \index{\& (cell separator)}
                \item{tabular environment}
                \item{l (column specifier)}
                \item{c (column specifier)}
                \item{r (column specifier)}
                \item{\textbackslash\textbackslash{} (row terminator)}
                \item{\& (cell separator)}
            \end{itemize}
        \end{framed}

        A simple \Lx{} table is created within a \texttt{tabular} environment. The column specifications are passed as a parameter to te environment --- r (right), c (center), l (left). Table rows are terminated by two backslashes (\textbackslash\textbackslash). Cells are separated by an ampersand (\&).

        \begin{tabular}{l c r}
Ann &  Derek &   Gladiola \\
Brenda &  Ed &   Hibiscus \\
Claudette &  Frederick &   Impatiens \\
        \end{tabular}


        \begin{verbatim}
\documentclass{article}
    \title{Smple Tables}
    \author{Charles Carter}
    \date{\today{}}
\begin{document} 
    \maketitle{}
        \begin{tabular}{l c r}
Ann         & Derek         & Gladiola \\
Brenda      & Ed            & Hibiscus \\
Claudette   & Frederick     & Impatiens \\
        \end{tabular}
\end{document}    
        \end{verbatim}

        \paragraph{Exercise:} Make your own simple table.

        \subsection{Row and column lines}
        \label{Row and column lines}
        
        \begin{framed}
            \begin{itemize}
                \index{\textbar}
                \index{hline}
                \item{\textbar}
                \item{hline}
            \end{itemize}
        \end{framed}

        Table rules are specified by the vertical bar (\textbar) and by \texttt{hline}. Column rules are set in the parameter of the column specifications passed to the tabular environment. Row rules are created by the \texttt{hline} command. Both of these can be doubled to produce double rules.
        

    \begin{tabular}{| l | r || l | r |}
        \hline
        Girls & Boys & Flowers & Animals\\
        \hline
        \hline
        Abigail &  Derek &   Gladiola & Jackel \\
        \hline
        Brenda &  Edgar &   Hibiscus & Koala \\
        \hline
        Claudia &  Frank &   Impatiens & Lynx \\
        \hline
    \end{tabular}


         \begin{verbatim}
\documentclass{article}
    \title{Table Rules}
    \author{Charles Carter}
    \date{\today{}}
\begin{document} 
    \maketitle{}
    \begin{tabular}{| l | r || l | r |}
        \hline
        Girls & Boys & Flowers & Animals\\
        \hline
        \hline
        Abigail &  Derek &   Gladiola & Jackel \\
        \hline
        Brenda &  Edgar &   Hibiscus & Koala \\
        \hline
        Claudia &  Frank &   Impatiens & Lynx \\
        \hline
    \end{tabular}
\end{document}    
        \end{verbatim}        

        \paragraph{Exercise:} Create your own simple table with vertical ane horizontal rules.

        \subsection{Column spacing}
        \label{Column spacing}
        
        \begin{framed}
            \begin{itemize}
                \index{p (column specifier)}
                \item{p (column specifier)}
            \end{itemize}
        \end{framed}

        The column specifiers \texttt{l} (left), \texttt{c} (center), and \texttt{}r (right) mostly do what you want, but their capacity is limited and frequently you need to use the column specifier \texttt{p} (paragraph) to make the content of the cells behave. \texttt{p} takes as a parameter a length recognized by \Lx{}, frequently a common English measure expressed in inches, for example, \texttt{p\{2in\}}.

        \begin{table}[!ht]
            \label{P columns}
        \begin{tabular}{ p{1in} p{4in}}
            \hline{}
            C & C is a general-purpose, imperative computer programming language, supporting structured programming, lexical variable scope and recursion, while a static type system prevents many unintended operations. \\
            Java & Java is a general-purpose computer programming language that is concurrent, class-based, object-oriented, and specifically designed to have as few implementation dependencies as possible. It is intended to let application developers ``write once, run anywhere'' meaning that compiled Java code can run on all platforms that support Java without the need for recompilation. \\
            Common Lisp & Common Lisp (historically, LISP) is a family of computer programming languages with a long history and a distinctive, fully parenthesized prefix notation. Originally specified in 1958, Lisp is the second-oldest high-level programming language in widespread use today. Only Fortran is older, by one year. \\
            JavaScript & JavaScript is a high-level, dynamic, untyped, and interpreted programming language. It has been standardized in the ECMAScript language specification. \\
            Erlang & Erlang is a programming language designed for developing robust systems of programs that can be distributed among different computers in a network. Named for the Danish mathematician Agner Krarup Erlang, the language was developed by the Ericsson Computer Sciences Lab to build software for its own telecommunication products.  \\
            \hline{}
        \end{tabular}
            \caption{P columns}
        \end{table}

        \begin{verbatim}
\documentclass{article}
    \title{This is My Title}
    \author{Charles Carter}
    \date{\today{}}
\begin{document} 
    \maketitle{}
        \begin{tabular}{ p{1in} p{4in}}
            \hline{}
            C & C is a general-purpose, imperative computer programming 
            language, supporting structured programming, lexical variable 
            scope and recursion, while a static type system prevents many 
            unintended operations. \\
            Java & Java is a general-purpose computer programming language 
            that is concurrent, class-based, object-oriented, and 
            specifically designed to have as few implementation 
            dependencies as possible. It is intended to let application 
            developers "write once, run anywhere" meaning that compiled 
            Java code can run on all platforms that support Java without 
            the need for recompilation. \\
            Common Lisp & Common Lisp (historically, LISP) is a family 
            of computer programming languages with a long history and a 
            distinctive, fully parenthesized prefix notation. Originally 
            specified in 1958, Lisp is the second-oldest high-level 
            programming language in widespread use today. Only Fortran 
            is older, by one year. \\
            JavaScript & JavaScript is a high-level, dynamic, untyped, 
            and interpreted programming language. It has been 
            standardized in the ECMAScript language specification. \\
            Erlang & Erlang is a programming language designed for 
            developing robust systems of programs that can be 
            distributed among different computers in a network. 
            Named for the Danish mathematician Agner Krarup Erlang, 
            the language was developed by the Ericsson Computer 
            Sciences Lab to build software for its own tele
            communication products.  \\
            \hline{}
        \end{tabular}
\end{document}    
        \end{verbatim}

        \paragraph{Exercise:} Create your own table with paragraph spacing.

        \subsection{Captions and labels}
        \label{Captions and labels}
        
        \begin{framed}
            \begin{itemize}
                \index{table environment}
                \index{caption}
                \index{label}
                \index{listoftables}
                \item{table environment}
                \item{caption}
                \item{label}
                \item{listoftables}
            \end{itemize}
        \end{framed}

        The \textit{table environment} is created with the \texttt{begin\{table\}} command, and ended with the \texttt{end\{table\}} command. Table environments contain tables (inside the \textit{tabular environment}), and optional labels and captions. A caption is created with the \texttt{caption} command. Notice that in the example below the label command is placed \textit{after} the caption command. Table \ref{Names} is on \pageref{Names} and table \ref{Plants} is on page \pageref{Plants}.

        The \texttt{listoftables} command, which should be placed just after the \texttt{tableofcontants} command, generates a list of the tables in the document. For an example, see the table of contents of this tutorial and the listing below.

    \begin{table}
    \caption{Names - Caption Above}
    \label{Names}
        \begin{tabular}{r | l l}
            \hline
            A & Akin & Allie \\
            B & Bobby & Bonita \\
            C & Chad & Carly \\
            \hline
        \end{tabular}
    \end{table}

    \begin{table}
        \begin{tabular}{r | l l}
            \hline
            D & Date & Dogwood  \\
            E & Eggplant & Elm \\
            F & Fig & Frasier Fir \\
            \hline
        \end{tabular}
    \caption{Plants - Caption Below}
    \label{Plants}
    \end{table}

        \begin{verbatim}
\documentclass{article}
    \title{Table Captions and Labels}
    \author{Charles Carter}
    \date{\today{}}
\begin{document} 
    \maketitle{}
    \listoftables{}
Reference to table \ref{Plants} on page \pageref{Plants}.
    \begin{table}
        \caption{Names - Caption Above}
        \label{Names}
        \begin{tabular}{r | l l}
            \hline
            A & Akin & Allie \\
            B & Bobby & Bonita \\
            C & Chad & Carly \\
            \hline
        \end{tabular}
    \end{table}
    \newpage
    \begin{table}
        \begin{tabular}{r | l l}
            \hline
            D & Date & Dogwood  \\
            E & Eggplant & Elm \\
            F & Fig & Frasier Fir \\
            \hline
        \end{tabular}
        \caption{Plants - Caption Below }
        \label{Plants}
    \end{table}
    Reference to table \ref{Names} on page \pageref{Names}.
\end{document}    
        \end{verbatim}

        \paragraph{Exercise:} You can refer to a table, and the page number of the table, by using the \texttt{ref} and \texttt{pageref} commands. See section \ref{Text references} on page \pageref{Text references}.

        \paragraph{Exercise:} How would you center a table horizontally on the page?



        \subsection{Multicells}
        \label{Multicells}
        
        \begin{framed}
            \begin{itemize}
                \index{multicolumn}
                \index{multirow package}
                \index{multirow}
                \item{multicolumn}
                \item{multirow package}
                \item{multirow}
            \end{itemize}
        \end{framed}

        \Lx{} tables can be nested, but for multi-column and multi-row spans, there are separate commands that are much easier to use. Multicolumn cells are created by the \texttt{multicolumn\{num-of-cols\}\{col-alignment\}\{cell-content\}} as illustrated in table \ref{multicol} on page \pageref{multicol}.

        \begin{table}
            \begin{tabular}{| l | p{2.5in} |  c |}
                \hline
                \multicolumn{3}{| c |}{Measures of center} \\
                \hline
                mean & the sum of all values divided by the number of values & $\frac{\sum_{i=1}^{i=n} x_i}{n}$ \\
                \hline
                median & the middle value in a vector of sorted value & $\frac{length(V) + 1}{n} | sorted(V)$  \\
                \hline
                mode & the most frequent value or values if more than one  & (no formula) \\
                \hline
                \multicolumn{3}{| c |}{Measures of dispersion} \\
                \hline
                variance & the sum of the difference between each value of the vector and the mean of the vector divided by the number of values in the vector & $\sum_{i=1}^{i=n} \frac{V_i - \mu}{n}$ \\
                \hline
                standard deviation & the square root of the sum of the difference between each value of the vector and the mean of the vector divided by the number of values in the vector  & $\sqrt{\sum_{i=1}^{i=n} \frac{V_i - \mu}{n}}$ \\
                \hline
                median absolute deviation & the average distance between each data value and the mean & $\frac{\sum_{i=1}^{i=n}|V_i - \mu|}{n}$  \\ 
                \hline
            \end{tabular}
        \caption{Multicolumn Example}
        \label{multicol}
        \end{table}

        \begin{verbatim}
\documentclass{article}
    \title{This is My Title}
    \author{Charles Carter}
    \date{\today{}}
\begin{document} 
    \begin{table}
        \begin{tabular}{| l | p{2.5in} |  c |}
            \hline
            \multicolumn{3}{| c |}{Measures of center} \\
            \hline
            mean & 
                the sum of all values divided by the number of values & 
                $\frac{\sum_{i=1}^{i=n} x_i}{n}$ \\
            \hline
            median & 
                the middle value in a vector of sorted value & 
                $\frac{length(V) + 1}{n} | sorted(V)$  \\
            \hline
            mode & 
                the most frequent value or values if more than one  
                & (no formula)  \\
            \hline
            \multicolumn{3}{| c |}{Measures of dispersion} \\
            \hline
            variance & 
                the sum of the difference between each value of the vector 
                and the mean of the vector divided by the number of values 
                in the vector & 
                $\sum_{i=1}^{i=n} \frac{V_i - \mu}{n}$ \\
            \hline
            standard deviation & 
            the square root of the sum of the difference between each value 
            of the vector and the mean of the vector divided by the number 
            of values in the vector  & 
            $\sqrt{\sum_{i=1}^{i=n} \frac{V_i - \mu}{n}}$ \\
            \hline
            median absolute deviation & the average distance between each data value and the mean & $\frac{\sum_{i=1}^{i=n}|V_i - \mu|}{n}$  \\ 
            \hline
        \end{tabular}
    \caption{Multicolumn Example}
    \label{multicol}
    \end{table}
\end{document}    
        \end{verbatim}

        \paragraph{Exercise:}

        \paragraph{Exercise:}

        \subsection{Long tables}
        \label{Long tables}
        
        \begin{framed}
            \begin{itemize}
                \item{}
            \end{itemize}
        \end{framed}


        \begin{verbatim}
\documentclass{article}
    \title{This is My Title}
    \author{Charles Carter}
    \date{\today{}}
\begin{document} 
    \maketitle{}
    \section{Introduction}
    \label{Introduction}
    \section{Body}
    \label{Body}
    \section{Conclusion}
    \label{Conclusion}
\end{document}    
        \end{verbatim}

        \paragraph{Exercise:}

        \paragraph{Exercise:}



    \section*{\LaTeXe{} is \textit{Free}}

    %images.tex
	
	\section{Images}
	\label{Images}

        \subsection{Image types}
        \label{Image types}
        
        \begin{framed}
            \begin{itemize}
                \item{}
            \end{itemize}
        \end{framed}


        \begin{verbatim}
\documentclass{article}
    \title{This is My Title}
    \author{Charles Carter}
    \date{\today{}}
\begin{document} 
    \maketitle{}
    \section{Introduction}
    \label{Introduction}
    \section{Body}
    \label{Body}
    \section{Conclusion}
    \label{Conclusion}
\end{document}    
        \end{verbatim}

        \paragraph{Exercise:}

        \paragraph{Exercise:}

        \subsection{Including graphics}
        \label{Including graphics}
        
        \begin{framed}
            \begin{itemize}
                \item{}
            \end{itemize}
        \end{framed}


        \begin{verbatim}
\documentclass{article}
    \title{This is My Title}
    \author{Charles Carter}
    \date{\today{}}
\begin{document} 
    \maketitle{}
    \section{Introduction}
    \label{Introduction}
    \section{Body}
    \label{Body}
    \section{Conclusion}
    \label{Conclusion}
\end{document}    
        \end{verbatim}

        \paragraph{Exercise:}

        \paragraph{Exercise:}

        \subsection{Image manipulation}
        \label{}
        
        \begin{framed}
            \begin{itemize}
                \item{}
            \end{itemize}
        \end{framed}


        \begin{verbatim}
\documentclass{article}
    \title{This is My Title}
    \author{Charles Carter}
    \date{\today{}}
\begin{document} 
    \maketitle{}
    \section{Introduction}
    \label{Introduction}
    \section{Body}
    \label{Body}
    \section{Conclusion}
    \label{Conclusion}
\end{document}    
        \end{verbatim}

        \paragraph{Exercise:}

        \paragraph{Exercise:}

        \subsection{}
        \label{}
        
        \begin{framed}
            \begin{itemize}
                \item{}
            \end{itemize}
        \end{framed}


        \begin{verbatim}
\documentclass{article}
    \title{This is My Title}
    \author{Charles Carter}
    \date{\today{}}
\begin{document} 
    \maketitle{}
    \section{Introduction}
    \label{Introduction}
    \section{Body}
    \label{Body}
    \section{Conclusion}
    \label{Conclusion}
\end{document}    
        \end{verbatim}

        \paragraph{Exercise:}

        \paragraph{Exercise:}

        \subsection{Captions and labels}
        \label{Captions and labels}
        
        \begin{framed}
            \begin{itemize}
                \item{}
            \end{itemize}
        \end{framed}


        \begin{verbatim}
\documentclass{article}
    \title{This is My Title}
    \author{Charles Carter}
    \date{\today{}}
\begin{document} 
    \maketitle{}
    \section{Introduction}
    \label{Introduction}
    \section{Body}
    \label{Body}
    \section{Conclusion}
    \label{Conclusion}
\end{document}    
        \end{verbatim}

        \paragraph{Exercise:}

        \paragraph{Exercise:}

        \subsection{Colors}
        \label{Colors}
        
        \begin{framed}
            \begin{itemize}
                \item{}
            \end{itemize}
        \end{framed}


        \begin{verbatim}
\documentclass{article}
    \title{This is My Title}
    \author{Charles Carter}
    \date{\today{}}
\begin{document} 
    \maketitle{}
    \section{Introduction}
    \label{Introduction}
    \section{Body}
    \label{Body}
    \section{Conclusion}
    \label{Conclusion}
\end{document}    
        \end{verbatim}

        \paragraph{Exercise:}

        \paragraph{Exercise:}




    %advanced.tex
	
	\section{Advanced Topics}
	\label{Advanced Topics}

    This part of the tutorial covers features that, while not really advanced, illustrate useful applications of \Lx{}. Section \ref{New commands} shows how to create new \Lx{} commands that can be useful in automating certain tasks. Section \ref{Templates} addresses templates that allow authors to conform to certainstyle conventions, such as MLA, APA, and IEEE. Section \ref{Source listing} concerns the reproduction of this document using the source listing. Sections \ref{Presentations} and \ref{Books} extend the \Lx{} document classes to books and presentations. Section \ref{Geometry} explorese a very useful package for adjusting page margins. Finally, section \ref{Auxiliary files} discusses certain files \Lx{} produces during the compilation process.

        \subsection{New commands}
        \label{New commands}
        
        \begin{framed}
            \begin{itemize}
                \item{}
            \end{itemize}
        \end{framed}


        \begin{verbatim}
\documentclass{article}
    \title{This is My Title}
    \author{Charles Carter}
    \date{\today{}}
\begin{document} 
    \maketitle{}
    \section{Introduction}
    \label{Introduction}
    \section{Body}
    \label{Body}
    \section{Conclusion}
    \label{Conclusion}
\end{document}    
        \end{verbatim}

        \paragraph{Exercise:}

        \paragraph{Exercise:}


        \subsection{Templates}
        \label{Templates}
        
        \begin{framed}
            \begin{itemize}
                \item{}
            \end{itemize}
        \end{framed}


        \begin{verbatim}
\documentclass{article}
    \title{This is My Title}
    \author{Charles Carter}
    \date{\today{}}
\begin{document} 
    \maketitle{}
    \section{Introduction}
    \label{Introduction}
    \section{Body}
    \label{Body}
    \section{Conclusion}
    \label{Conclusion}
\end{document}    
        \end{verbatim}

        \paragraph{Exercise:}

        \paragraph{Exercise:}


        \subsection{Geometry}
        \label{Geometry}
        
        \begin{framed}
            \begin{itemize}
                \item{}
            \end{itemize}
        \end{framed}


        \begin{verbatim}
\documentclass{article}
    \title{This is My Title}
    \author{Charles Carter}
    \date{\today{}}
\begin{document} 
    \maketitle{}
    \section{Introduction}
    \label{Introduction}
    \section{Body}
    \label{Body}
    \section{Conclusion}
    \label{Conclusion}
\end{document}    
        \end{verbatim}

        \paragraph{Exercise:}

        \paragraph{Exercise:}


        \subsection{Includes}
        \label{Includes}
        
        \begin{framed}
            \begin{itemize}
                \item{}
            \end{itemize}
        \end{framed}


        \begin{verbatim}
\documentclass{article}
    \title{This is My Title}
    \author{Charles Carter}
    \date{\today{}}
\begin{document} 
    \maketitle{}
    \section{Introduction}
    \label{Introduction}
    \section{Body}
    \label{Body}
    \section{Conclusion}
    \label{Conclusion}
\end{document}    
        \end{verbatim}

        \paragraph{Exercise:}

        \paragraph{Exercise:}


        \subsection{Source listing}
        \label{Source listing}
        
        \begin{framed}
            \begin{itemize}
                \item{}
            \end{itemize}
        \end{framed}


        \begin{verbatim}
\documentclass{article}
    \title{This is My Title}
    \author{Charles Carter}
    \date{\today{}}
\begin{document} 
    \maketitle{}
    \section{Introduction}
    \label{Introduction}
    \section{Body}
    \label{Body}
    \section{Conclusion}
    \label{Conclusion}
\end{document}    
        \end{verbatim}

        \paragraph{Exercise:}

        \paragraph{Exercise:}


        \subsection{Presentations}
        \label{Presentations}
        
        \begin{framed}
            \begin{itemize}
                \index{beamer documentclass}
                \index{documentclass beamer}
                \index{frame}
                \index{frametitle}
                \item{beamer documentclass}
                \item{frame}
                \item{frametitle}
            \end{itemize}
        \end{framed}

        \Lx{} creates excellent slide presentations. The following example illustrates the use of the \textit{beamer} package. Beamer has a great deal of flexibility, and it has the advantage that you can recycle your \Lx{} code that you used for your document to create your presentation. This short tutorial cannot covert beamer at all but it can suggest how beamer can be used to produce a presentation. 

        \begin{realverbatim}
\documentclass{beamer}
    \title{Beamer Example}
    \author{Charles Carter}
    \date{\today}
\begin{document}
    \begin{frame}
        \maketitle
    \end{frame}
    \begin{frame}
        \frametitle{Table of Contents}
        \tableofcontents{}
    \end{frame}
    \section{Lists}
    \begin{frame}
        \frametitle{Lists}
        This section covers lists, unnumbered, numbered, and dictionary lists.
    \end{frame}
    \subsection{Unnumbered lists}
    \begin{frame}
        \frametitle{Unnumbered lists}
        \begin{itemize}
            \item{apples}
            \item{bananas}
            \item{cranberries}
            \item{dates}
        \end{itemize}
    \end{frame}
    \subsection{Paused unnumbered lists}
    \begin{frame}
        \frametitle{Unnumbered lists with pause}
        \begin{itemize}
            \item{endive}
            \pause
            \item{figs}
            \pause
            \item{garlic} 
            \pause
            \item{honeydew melon}
        \end{itemize}
    \end{frame}
    \subsection{Numbered lists}
    \begin{frame}
        \frametitle{Numbered lists}
        \begin{enumerate}
            \item{iceberg lettuce}
            \item{Jerusalem artichoke}
            \item{kiwi}
            \item{lime}
        \end{enumerate}
    \end{frame}
    \subsection{Paused numbered lists}
    \begin{frame}
        \frametitle{Numbered lists with pause}
        \begin{enumerate}
            \item{mango}
            \pause
            \item{nectarine}
            \pause
            \item{olive} 
            \pause
            \item{paw paw}
        \end{enumerate}
    \end{frame}
    \subsection{Dictionary lists}
    \begin{frame}
        \frametitle{Dictionary lists}
        \begin{description}
            \item[quince]{cooked in pies}
            \item[radish]{raw in salads}
            \item[strawberry]{fresh as a dessert}
            \item[turnip]{raw in salad or cooked with greens}
        \end{description}
    \end{frame}
    \subsection{Paused dictionary lists}
    \begin{frame}
        \frametitle{Numbered lists with pause}
        \begin{description}
            \item[ugli fruit]{fresh at breakfast}
            \pause
            \item[watermelon]{at picnics}
            \pause
            \item[yam]{baked with turkey or ham}
            \pause
            \item[zucchini]{boiled or stir fried}
        \end{description}
    \end{frame}
    \section{Tables, blocks, and verbatim}
    \begin{frame}
        \frametitle{Tables, blocks, and verbatim}
        This section covers tables, blocks, and listings using the \texttt{verbatim} environment.
    \end{frame}
    \subsection{Tables}
    \begin{frame}
        \frametitle{Tables}
        \begin{tabular}{| l | l | l |} 
            \hline
            Beef & steak & pot roast \\
            \hline
            Pork & bacon & spare ribs \\
            \hline
            Fowl & fried chicken & chicken livers \\
            \hline
        \end{tabular}
    \end{frame}
    \subsection{Blocks}
    \begin{frame}
        \frametitle{Blocks}
        \begin{block}{Beef stroganoff} 
            contains both milk and meat and is not kosher.
        \end{block}
        \begin{block}{Pork products} 
            such as wieners and BBQ, are not kosher
        \end{block}
        \begin{block}{Shellfish} 
            such as oysters, shrimp, and clams, is not kosher.
        \end{block}
    \end{frame}
    \subsection{Verbatim}
    \begin{frame}[containsverbatim]
        \frametitle{Verbatim}
        Here is the listing from the previous slide.
        \begin{verbatim}
    \begin{block}{Beef stroganoff} 
        contains both milk and meat and is not kosher.
    \end{block}
    \begin{block}{Pork products} 
        such as wieners and BBQ, are not kosher
    \end{block}
    \begin{block}{Shellfish} 
        such as oysters, shrimp, and clams, is not kosher.
    \end{block}
        \end{verbatim}
    \end{frame}
    \section{Afterword}
    \begin{frame}
        \frametitle{Afterword}
        This just scratches the surface. Beamer has much, much more functionality than illustrated here. If you are interested in Beamer, read the documentation for the details.
    \end{frame}
\end{document}
        \end{realverbatim}

        \paragraph{Exercise:} Write your own, very simple, presentation, compile it, and present it.


        \subsection{Books}
        \label{Books}
        
        \begin{framed}
            \begin{itemize}
                \index{book documentclass}
                \index{documentclass book}
                \item{book documentclass}
            \end{itemize}
        \end{framed}

        You can and should use \Lx{} for writing book-sized documents. Use the documentclass \texttt{book}. Here is a very brief example. If you compile this and open it, you will have the skeleton of a ``real book.'' As always, the book documentclass contains a number of particular commands and environments, and if you go to the effort of authoring a book, you should also go to the effort of familiarizing yourself with this document class. There are also other document classes, such as memoir, which are current more popular, and you may benefit from exploring these as well.

        \begin{verbatim}
\documentclass[letterpaper]{book}
	\title{My First Book}
	\author{Charles Carter}
	\date{\today{}}
\begin{document}
\frontmatter
\maketitle
\tableofcontents{}
\chapter{Preface}
This  is the preface
\mainmatter
    \chapter{A Story}
    Let me tell you a story.
    \chapter{Moral}
    This is the moral of the story. 
\appendix
    \chapter{First Appendix}
    Here is an appendix.
\backmatter
    \chapter{Last note}
    The last word.
\end{document}    
        \end{verbatim}

        \paragraph{Exercise:} Write your own book. Just kidding! However, you can make the skeleton of a very simple, short book and you may even find this useful for a very long paper.

        \subsection{Auxiliary files}
        \label{Auxiliary files}
        
        When you compile a \Lx{} document, the compilation process produces a number of auxiliary files. I have listed those produced by this tutorial below. You can open and read these in any text processor. You \textit{should} open and read the \texttt{log} file. Often, when you have a warning or error in the compilation process, you can find valuable information here. Also, if you know what you are doing, you can edit an appropriate auxiliary file to fine tune your document. In any case, you should not regard these auxiliary files as a mystery, but should become comfortable with the process that produces them.

        \begin{verbatim}
08/12/2016  01:10 PM             1,718 latex-tutorial.aux
08/04/2016  11:06 AM             2,388 latex-tutorial.bbl
08/04/2016  11:06 AM               662 latex-tutorial.blg
08/12/2016  01:10 PM               611 latex-tutorial.ent
08/12/2016  01:10 PM             3,732 latex-tutorial.idx
08/12/2016  01:10 PM               362 latex-tutorial.ilg
08/12/2016  01:10 PM             3,071 latex-tutorial.ind
08/12/2016  01:10 PM                 0 latex-tutorial.lof
08/12/2016  01:10 PM            50,555 latex-tutorial.log
08/12/2016  01:10 PM               586 latex-tutorial.lot
08/12/2016  01:10 PM             3,988 latex-tutorial.out
08/12/2016  01:10 PM           626,626 latex-tutorial.pdf
08/10/2016  07:35 PM             6,768 latex-tutorial.tex
08/12/2016  01:10 PM             9,723 latex-tutorial.toc
        \end{verbatim}

        \paragraph{Exercise:} Read the log file, all of it. Understand as much of it as you can.


    \section{Conclusion}
    \label{Conclusion}

    \appendix{}
    %appendix.tex


    \section{Installing \LaTeXe}
    \label{Installing}

    \Lx{} formally consists of macros built on \TeX{}, a typesetting program created by Donald Knuth in the late 1970s. \Lx{} itself was create by Leslie Lamport in the early 1980s. Both \TeX{} and \Lx{} are open technologies, which means that anyone can implement them without restriction. Several implementations of \Lx{} are available, and opinions differ as to whichn one is ``best.'' I have tried several different implementations, and on Microsoft Windows, MiKTeX works best for me. You mileage may vary.
    
    This tutorial uses the MikTeX implementation of \Lx{}. MiKTeX's home (as of Auust, 2016) is \url{http://miktex.org/}. It can be downloaded by following the Download link (figure \ref{img:MiKTeX download}).  Typically, accepting the usual defaults will lead to a successful installation. In particular, the installation process will add the executable to your path, so youcan invoke kit without jumping through any hoops.

    If all goes well (and it should) you will see MiKTeX in your Windows start menu, as in figure \ref{img:MiKTeX winstart}. Note the program \texttt{TeXworks} --- you can use this as your \Lx{} development environment if you want, see appendix \ref{Development Environments}. You can also test your installation by executing the command \texttt{pdflatex --help} using your command line interface, see figure \ref{img:CLI test}.

    \begin{figure}[!hb]
        \includegraphics[width=3in]{images/miktex-download.pdf}
        \caption{MiKTeX download}
        \label{img:MiKTeX download}
    \end{figure}

    \begin{figure}[!hb]
        \includegraphics[width=1in]{images/miktex-winstart.pdf}
        \caption{MiKTeX on Windows start menu}
        \label{img:MiKTeX winstart}
    \end{figure}

    \begin{figure}[!hb]
        \includegraphics[width=3in]{images/install-test.pdf}
        \caption{CLI test}
        \label{img:CLI test}
    \end{figure}

    \section{\LaTeXe Editors}
    \label{Editors}

    A \texttt{tex} file is just ordinary plain text, without any special codes or binary commands. You write your \Lx{} document using an ordinary text editor. You can use your favorite text editor with no restrictions, from Windows Notepad (the most minimal text editor I know) on up.\footnote{But be careful of Notepad. Notepad by default saves documents with the \texttt{txt} extension, i.e., as a plain text file. You \textit{must} save your \Lx{} file with a \texttt{tex} extension. Be sure to select the All Files option in the Save As box, and explicitly add \texttt{.tex} to the name of your file}

    Notepad++ is a very popular, full featured, free text editor, and I recommend it for a large number of applications, including \Lx{}. You can find Notepad++ at \url{https://notepad-plus-plus.org/}. Notepad++ is just a text editor, so it doesn't come with any special bells and whistles.\footnote{If you really want to add a compilation feature to Notepad++, search the internet. It's not hard, but I don't cover it here.} Personally, I use \texttt{gvim} as my text editor, as it works for virtually everything, and works on Windows, Mac, Linux, and UNIX. However, \texttt{gvim} requires a substantial investment of time and effort to achieve proficiency, and I don't cover it here.

    Typically, a new user of \Lx{} will choose to use an integrated development which combines a text editor and compiler (see appendix \ref{Development Environments}) or an online resource (see appendix \ref{Online}). Use of a text editor requires separate editing and compilation cycles, which does offer some advantages, which is why some people choose to do it this way.

    \section{Development Environments}
    \label{Development Environments}

    MiKTeX comes with TeXwriter, which combines a text editor with a compiler. It's pretty minimal and crude, but that means that there's less to learn and less to break. Many other \Lx{} programs can be had, both free and commercial. Look at the Wikipedia entry on \Lx{} or \TeX{} editors. TeXmaker is also popular. The advantage to a development environment is that you can do your writing, editing, compilation, and publishing with one application. These also offer nice features, such as syntax coloring, autocomplete, and builtin help systems. The top end development environments also have point and click, drag and drop, features. This results in something close to a blend of a word processor (like Microsoft Word) and \Lx{}. Whether or not this is a good thing is a matter of opinion. You will find that people who use \Lx{} professionaly do not use these kinds of graphical interfaces --- there are very good reasons for this avoidance that you will come to see if you use \Lx{} frequently.

    \section{Online \LaTeXe}
    \label{Online}

    You can also use online \Lx{} editors. There are many equation editors that translate equations into \Lx{} source code. There are also full fledged online \Lx{} systems, which alalow collaboration between colleages. Many (virtually all) are commercial. I have never used any of these so I can't offer an opinion about them. If you do not have an adequate computer (such as a Chromebook, perhaps), online document writing is an option.

    \section{Command Line Execution}
    \label{Command Line}

    Both \TeX{} and \Lx{} consist of executable computer programs that can be invoked and executed on the command line. If you automate your processes (perhaps you have to create lengthy reports several times a day or a week) you will write programs that call these executables to produce your documents. The three commands you will use most often are \texttt{pdflatex}, \texttt{makeindex}, and \texttt{bibtex}. Power users commonly do their work on the command line, and again, there are good reasons for this.

    If the thought of the command line terrifies you, you never have to see it. However, it's always there lurking in the background. And sometimes, use of the command line is the only way to perform some tasks that your GUI program does not implement.




    \theendnotes{}
    \addcontentsline{toc}{section}{Endnotes}

    \bibliographystyle{IEEEtranS}    
    \bibliography{tut}
    \addcontentsline{toc}{section}{References}
    \nocite{*}

    \addcontentsline{toc}{section}{Index}
    \printindex{}

\end{document}

%        \subsection{}
%        \label{}
%        
%        \begin{framed}
%            \begin{itemize}
%                \item{}
%            \end{itemize}
%        \end{framed}
%
%
%        \begin{verbatim}
%\documentclass{article}
%    \title{This is My Title}
%    \author{Charles Carter}
%    \date{\today{}}
%\begin{document} 
%    \maketitle{}
%    \section{Introduction}
%    \label{Introduction}
%    \section{Body}
%    \label{Body}
%    \section{Conclusion}
%    \label{Conclusion}
%\end{document}    
%        \end{verbatim}
%
%        \paragraph{Exercise:}
%
%        \paragraph{Exercise:}


