%references.tex
    
    \section{References}
    \label{References}

    In scientific and research writing, one of the most critical aspects is references. Think about it: can you imagine a research paper without a single reference? These include footnotes, endnotes, marginal notes, and especially citations to sources. References have two components, a target and a source, or a label and a reference to that label. We have used labels before, but now we will expicitly consider them.

    \subsection{Footnotes}
    \label{Footnotes}

        \begin{framed}
            \begin{itemize}
                \item{footnote}
                \item{label}
                \item{ref}
                \item{pageref}
            \end{itemize}
        \end{framed}

    To insert a footnote, just use the command \text{footnote}.\footnote{\label{references:fn}Don't forget to put the \textbackslash{} before the \texttt{footnote} command.} If you label the footnote, you can refer to the footnote by number and page in the text of the doucumemt. Please be sure to read footnote \ref{references:fn} on page \pageref{references:fn}.

        \begin{verbatim}
\documentclass{article}
    \title{Footnotes and References}
    \author{Charles Carter}
    \date{\today{}}
\begin{document} 
    \maketitle{}
    To insert a footnote, just use the command \text{footnote}.\footnote{\label{references:fn}
    Don't forget to put the \textbackslash{} before the \texttt{footnote} command.} If you 
    label the footnote, you can refer to the footnote by number and page in the text of the
    doucumemt. Please be sure to read footnote \ref{references:fn} on page 
    \pageref{references:fn}.
\end{document}    
        \end{verbatim}

        \paragraph{Exercise:} Read the \LaTeXe{} documentation on footnotes.

    \subsection{Endnotes}
    \label{Endnotes}
        
        \begin{framed}
            \begin{itemize}
                \item{package endnotes}
                \item{endnote}
                \item{theendnotes}
                \item{addcontentsline}
            \end{itemize}
        \end{framed}

    Endnotes are a little more complicated than footnotes, but not much.\endnote{The difference between footnotes and endnotes is that footnotes go at the foot of the page where they appear, whilc endnotes appear at the end of the document.} Here is an endnote.\endnote{This is an endnote.} In order to actually print the endnotes, use the \texttt{theendnotes} command. In order to create an entry for the endnotes in the table of contents, you must use the \texttt{addcontentsline}.\endnote{\label{references:en}The addcontentsline takes three parameters, where the line should be written, usually \textit{toc}, the formatting to be used, usually \textit{section}, and the name to be given to the entry, perhaps \textit{Endnotes}.} Please see endnote \ref{references:en} on page \pageref{references:en} for the details.

        \begin{verbatim}
\documentclass{article}
    \usepackage{endnotes}
    \title{Endnotes}
    \author{Charles Carter}
    \date{\today{}}
\begin{document} 
    \maketitle{}
    \tableofcontents{}
    \section{Text}
    Endnotes are a little more complicated than footnotes, but not much.\endnote{The 
    difference between footnotes and endnotes is that footnotes go at the foot of 
    the page where they appear, whilc endnotes appear at the end of the document.} 
    Here is an endnote.\endnote{This is an endnote.} In order to actually print 
    the endnotes, use the \texttt{theendnotes} command. In order to create an 
    entry for the endnotes in the table of contents, you must use the \texttt{addcontentsline}.
    \endnote{\label{references:en}The addcontentsline takes three parameters, 
    where the line should be written, usually \textit{toc}, the formatting to be 
    used, usually \textit{section}, and the name to be given to the entry, perhaps 
    \textit{Endnotes}.} Please see endnote \ref{references:en} on page 
    \pageref{references:en} for the details.
    \theendnotes{}
    \addcontentsline{toc}{section}{Endnotes}
\end{document}    
        \end{verbatim}

        \paragraph{Exercise:} Find and read through the documentation of the Endnotes package.

    \subsection{Margin notes}
    \label{Margin notes}

        \begin{framed}
            \begin{itemize}
                \item{marginpar}
                \item{reversemarginpar}
                \item{normalmarginpar}
                \item{raggedright}
            \end{itemize}
        \end{framed}

        \marginpar{Important point!}

        Marginal notes are really useful to call attention to very important points.  They can also be useful in drafting documents to note future edits, insertions, or deletions. Use \texttt{marginpar} to create a marginal note. To delete marginal notes during the drafting process, just comment them out. The text remains as a reminder of the modifications in the document.

        {\raggedright\reversemarginpar{\marginpar{This is reversed text}}}

        To permanantly reverse the page sides. use \texttt{reversemarginpar}. To reverse the reverse page margins, use \texttt{normalmarginpar}. To change the paragraph alignment from the default justified text, use the command \texttt{raggedright} in a block.

        {\raggedright\normalmarginpar\marginpar{This is in the correct margim}}

        \begin{verbatim}
\documentclass{article}
    \title{Margin Notes}
    \author{Charles Carter}
    \date{\today{}}
\begin{document} 
    \maketitle{}
        \marginpar{Important point!}
        Marginal notes are really useful to call attention to very important 
        points.  They can also be useful in drafting documents to note 
        future edits, insertions, or deletions. Use \texttt{marginpar} 
        to create a marginal note. To delete marginal notes during the 
        drafting process, just comment them out. The text remains as a 
        reminder of the modifications in the document.
        {\raggedright\reversemarginpar{\marginpar{This is reversed text}}}
        To permanantly reverse the page sides. use \texttt{reversemarginpar}.
        To reverse the reverse page margins, use \texttt{normalmarginpar}.
        {\raggedright\normalmarginpar\marginpar{This is in the correct margim}}
\end{document}    
        \end{verbatim}

        \paragraph{Exercise:} If you really need to create sophisticated marginal notes, you will need both the \texttt{geometry} and \texttt{marginnote} packages. The former is very useful, and you may need to use it in every paper. The latter is useful for marginal notes when you need something more than the default \LaTeXe{} commands.

    \subsection{Bibliography}
    \label{Bibliography}

        \begin{framed}
            \begin{itemize}
                \item{creating an external \bib{} database}
            \end{itemize}
        \end{framed}

    Needless to say, every research paper requires a bibliography. \LaTeXe{} has multiple ways to include bibliographies. I have chosen to use a method called \bib{}. There are three reasons for this choice. First. it's simple enough to include in a simple tutorial. Second, it gives good results using its default settings. Third, it's complex enough to be configured for almost any application (given an author's time and patience).

    \bib{} uses an external file as a bibliographical database. This means that management of the sources is separate from writing and editing the original document. The data from the external database is meerged into the original document with a series of commands. We will cover the basics of the external database in this lesson. In the following lesson, we will see how to merge the two files.

    The external \bib{} database is merely a plain text file with entries in a prescribed format. It \textit{must} have the \texttt{.bib} extension for the filename. The following file includes four books and three online sources. The \bib format file contains specifications for a large number of types of sources, and you should become familiar with the various types of documents and the format for each.    

    Create a document with the following content. Name it \texttt{tut.bib} and place it in the same directory with your original source document, where you compile your PDF output. Each entry has a type identifier followed by a curly brace pair (\{\ldots{}\}), within which are contained various fields separated by commas. The first field is the identifier you will use to access the entry. The subsequent fields are ``key''=``value'' pairs, giving the title, author, date etc.
    
        \begin{verbatim}
\\tut.bib
@book{
    goossens04,
    author    =  {Frank Mittelbach and 
        Michel Goossens and 
        Johannes Braams and 
        David Carlisle and 
        Chris Rowley},
    title     = {The \LaTeX{} Companion (Tools and Techniques for Computer Typesetting)},
    year      = {2004},
    edition =2nd},
    publisher = {Addison-Wesley},
    address   = {Reading, MA}
    ISBN      = {978-0201362992}
}
@book{
    kottwitz11,
    author    = {Stefan Kottwitz },
    title     = {\LaTeX{} Beginner's Guide},
    year      = {2011},
    publisher = {Packt Publishing},
    address   = {Birmingham, UK},
    ISBN      = {978-1847199867}
}
@book{
    kottwitz15,
    author    = {Stefan Kottwitz },
    title     = {\LaTeX{} Cookbook},
    year      = {2015},
    publisher = {Packt Publishing},
    address   = {Birmingham, UK},
    ISBN      = { 978-1784395148}
}
@book{
    gratzer14,
    author    = {George Gr{\''a}tzer},
    title     = {Practical \LaTeX{}},
    year      = {2014},
    publisher = {Springer},
    address   = {New York, NY},
    ISBN      = {978-1847199867}
}
@misc{oetiker15,
  author = {Tobias Oetiker and Hubert Partl and Irene Hyna and Elisabeth Schlegl},
  title = {The Not So Short Introduction to \LaTeXe{}},
  howpublished = "\url{http://tug.ctan.org/info/lshort/english/lshort.pdf}",
  year = {2015}, 
  note = "[Online; accessed August 2, 2016]"
}
@misc{pakin15,
  author = {Scott Pakin},
  title = {PThe Comprehensive LATEX Symbol List},
  howpublished = "\url{http://tug.ctan.org/info/symbols/comprehensive/symbols-letter.pdf}",
  year = {2015}, 
  note = "[Online; accessed August 2, 2016]"
}
@misc{carter16,
  author = {Charles Carter},
  title = {Another Short Tutorial to \LaTeXe{}},
  howpublished = "\url{https://github.com/ccc31807/latex-short-intro}",
  year = {2016}, 
  note = "[Online; accessed August 2, 2016]"
}
        \end{verbatim}

        \paragraph{Exercise:} Search for the \bib{} format specification. Name four different kinds of source documents and identifies the required elements for each.

    \subsection{Citations}
    \label{Citations}

%        \subsection{}
%        \label{}
%        
%        \begin{framed}
%            \begin{itemize}
%                \item{}
%            \end{itemize}
%        \end{framed}
%
%
%        \begin{verbatim}
%\documentclass{article}
%    \title{This is My Title}
%    \author{Charles Carter}
%    \date{\today{}}
%\begin{document} 
%    \maketitle{}
%    \section{Introduction}
%    \label{Introduction}
%    \section{Body}
%    \label{Body}
%    \section{Conclusion}
%    \label{Conclusion}
%\end{document}    
%        \end{verbatim}
%
%        \paragraph{Exercise:}
%
%        \paragraph{Exercise:}

    \subsection{Indices}
    \label{Indices}

%        \subsection{}
%        \label{}
%        
%        \begin{framed}
%            \begin{itemize}
%                \item{}
%            \end{itemize}
%        \end{framed}
%
%
%        \begin{verbatim}
%\documentclass{article}
%    \title{This is My Title}
%    \author{Charles Carter}
%    \date{\today{}}
%\begin{document} 
%    \maketitle{}
%    \section{Introduction}
%    \label{Introduction}
%    \section{Body}
%    \label{Body}
%    \section{Conclusion}
%    \label{Conclusion}
%\end{document}    
%        \end{verbatim}
%
%        \paragraph{Exercise:}
%
%        \paragraph{Exercise:}


