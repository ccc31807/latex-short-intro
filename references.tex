%references.tex
	
	\section{References}
	\label{References}

	In scientific and research writing, one of the most critical aspects is references. Think about it: can you imagine a research paper without a single reference? These include footnotes, endnotes, marginal notes, and especially citations to sources. References have two components, a target and a source, or a label and a reference to that label. We have used labels before, but now we will expicitly consider them.

	\subsection{Footnotes}
	\label{Footnotes}

        \begin{framed}
            \begin{itemize}
                \item{footnote}
                \item{label}
                \item{ref}
                \item{pageref}
            \end{itemize}
        \end{framed}

	To insert a footnote, just use the command \text{footnote}.\footnote{\label{references:fn}Don't forget to put the \textbackslash{} before the \texttt{footnote} command.} If you label the footnote, you can refer to the footnote by number and page in the text of the doucumemt. Please be sure to read footnote \ref{references:fn} on page \pageref{references:fn}.

        \begin{verbatim}
\documentclass{article}
    \title{Footnotes and References}
    \author{Charles Carter}
    \date{\today{}}
\begin{document} 
    \maketitle{}
	To insert a footnote, just use the command \text{footnote}.\footnote{\label{references:fn}
	Don't forget to put the \textbackslash{} before the \texttt{footnote} command.} If you 
	label the footnote, you can refer to the footnote by number and page in the text of the
	doucumemt. Please be sure to read footnote \ref{references:fn} on page 
	\pageref{references:fn}.
\end{document}    
        \end{verbatim}

        \paragraph{Exercise:} Read the \LaTeXe{} documentation on footnotes.

	\subsection{Endnotes}
	\label{Endnotes}
        
        \begin{framed}
            \begin{itemize}
                \item{package endnotes}
                \item{endnote}
                \item{theendnotes}
                \item{addcontentsline}
            \end{itemize}
        \end{framed}

	Endnotes are a little more complicated than footnotes, but not much.\endnote{The difference between footnotes and endnotes is that footnotes go at the foot of the page where they appear, whilc endnotes appear at the end of the document.} Here is an endnote.\endnote{This is an endnote.} In order to actually print the endnotes, use the \texttt{theendnotes} command. In order to create an entry for the endnotes in the table of contents, you must use the \texttt{addcontentsline}.\endnote{\label{references:en}The addcontentsline takes three parameters, where the line should be written, usually \textit{toc}, the formatting to be used, usually \textit{section}, and the name to be given to the entry, perhaps \textit{Endnotes}.} Please see endnote \ref{references:en} on page \pageref{references:en} for the details.

        \begin{verbatim}
\documentclass{article}
	\usepackage{endnotes}
    \title{Endnotes}
    \author{Charles Carter}
    \date{\today{}}
\begin{document} 
    \maketitle{}
	\tableofcontents{}
	\section{Text}
	Endnotes are a little more complicated than footnotes, but not much.\endnote{The 
	difference between footnotes and endnotes is that footnotes go at the foot of 
	the page where they appear, whilc endnotes appear at the end of the document.} 
	Here is an endnote.\endnote{This is an endnote.} In order to actually print 
	the endnotes, use the \texttt{theendnotes} command. In order to create an 
	entry for the endnotes in the table of contents, you must use the \texttt{addcontentsline}.
	\endnote{\label{references:en}The addcontentsline takes three parameters, 
	where the line should be written, usually \textit{toc}, the formatting to be 
	used, usually \textit{section}, and the name to be given to the entry, perhaps 
	\textit{Endnotes}.} Please see endnote \ref{references:en} on page 
	\pageref{references:en} for the details.
	\theendnotes{}
	\addcontentsline{toc}{section}{Endnotes}
\end{document}    
        \end{verbatim}

        \paragraph{Exercise:} Find and read through the documentation of the Endnotes package.

	\subsection{Margin Notes}
	\label{Margin Notes}

%        \subsection{}
%        \label{}
%        
%        \begin{framed}
%            \begin{itemize}
%                \item{}
%            \end{itemize}
%        \end{framed}
%
%
%        \begin{verbatim}
%\documentclass{article}
%    \title{This is My Title}
%    \author{Charles Carter}
%    \date{\today{}}
%\begin{document} 
%    \maketitle{}
%    \section{Introduction}
%    \label{Introduction}
%    \section{Body}
%    \label{Body}
%    \section{Conclusion}
%    \label{Conclusion}
%\end{document}    
%        \end{verbatim}
%
%        \paragraph{Exercise:}
%
%        \paragraph{Exercise:}

	\subsection{Bibliography}
	\label{Bibliography}

%        \subsection{}
%        \label{}
%        
%        \begin{framed}
%            \begin{itemize}
%                \item{}
%            \end{itemize}
%        \end{framed}
%
%
%        \begin{verbatim}
%\documentclass{article}
%    \title{This is My Title}
%    \author{Charles Carter}
%    \date{\today{}}
%\begin{document} 
%    \maketitle{}
%    \section{Introduction}
%    \label{Introduction}
%    \section{Body}
%    \label{Body}
%    \section{Conclusion}
%    \label{Conclusion}
%\end{document}    
%        \end{verbatim}
%
%        \paragraph{Exercise:}
%
%        \paragraph{Exercise:}

	\subsection{Citations}
	\label{Citations}

%        \subsection{}
%        \label{}
%        
%        \begin{framed}
%            \begin{itemize}
%                \item{}
%            \end{itemize}
%        \end{framed}
%
%
%        \begin{verbatim}
%\documentclass{article}
%    \title{This is My Title}
%    \author{Charles Carter}
%    \date{\today{}}
%\begin{document} 
%    \maketitle{}
%    \section{Introduction}
%    \label{Introduction}
%    \section{Body}
%    \label{Body}
%    \section{Conclusion}
%    \label{Conclusion}
%\end{document}    
%        \end{verbatim}
%
%        \paragraph{Exercise:}
%
%        \paragraph{Exercise:}

	\subsection{Indices}
	\label{Indices}

%        \subsection{}
%        \label{}
%        
%        \begin{framed}
%            \begin{itemize}
%                \item{}
%            \end{itemize}
%        \end{framed}
%
%
%        \begin{verbatim}
%\documentclass{article}
%    \title{This is My Title}
%    \author{Charles Carter}
%    \date{\today{}}
%\begin{document} 
%    \maketitle{}
%    \section{Introduction}
%    \label{Introduction}
%    \section{Body}
%    \label{Body}
%    \section{Conclusion}
%    \label{Conclusion}
%\end{document}    
%        \end{verbatim}
%
%        \paragraph{Exercise:}
%
%        \paragraph{Exercise:}


