%lists.tex
	
	\section{Lists}
	\label{Lists}

        \subsection{Verbatim}
        \label{Verbatim}
        
        \begin{framed}
            \begin{itemize}
                \index{verbatim environment}
                \index{verbatim}
                \item{verbatim environment}
                \item{verbatim}
            \end{itemize}
        \end{framed}

    Perhaps the simplest and easiest way to make a list is by using the \texttt{verbatim} environment. Everything in the environment is printed as is, that is, \Lx{} commands are not evaluated.

        \begin{realverbatim}
\documentclass{article}
    \title{Verbtim Environment}
    \author{Charles Carter}
    \date{\today{}}
\begin{document} 
    \maketitle{}
    \section{Verbatim Environment}
    \begin{verbatim}
        \texttt{one}
        \textit{two}
        \textsf{three}
        \huge{aye}
        \tiny{bee}
        \section{see}
    \end{verbatim}
\end{document}    
        \end{realverbatim}

        \paragraph{Exercise:} The only command not permitted in the \texttt{verbatim} environment is the \texttt{\textbackslash{}end\{verbatim\}} command. The reason is that this command ends the verbatim environment. If you examine the source of this tutorial, you will see that I resorted to a trick to include this command. As an exercise, can you figure out the trick?

        \subsection{Unnnmbered Lists}
        \label{Unnnumbered Lists}
        
        \begin{framed}
            \begin{itemize}
                \index{itemize}
                \index{item}
                \item{itemize}
                \item{item}
            \end{itemize}
        \end{framed}

        \Lx{} unnumbered lists are very easy. Simply enclose the list in th \texttt{itemize} environment, and place each list item as an argument to the \texttt{item} command.

    \begin{itemize}
        \item{Green}
        \item{Eggs}
        \item{and}
        \item{Ham}
    \end{itemize}

        \begin{verbatim}
\documentclass{article}
    \title{Unnumbered List}
    \author{Charles Carter}
    \date{\today{}}
\begin{document} 
    \maketitle{}
    \begin{itemize}
        \item{Green}
        \item{Eggs}
        \item{and}
        \item{Ham}
    \end{itemize}
\end{document}    
        \end{verbatim}

        \paragraph{Exercise:} Make your own unnumbered list.

        \subsection{Numbered Lists}
        \label{Numbered Lists}
        
        \begin{framed}
            \begin{itemize}
                \index{enumerate}
                \index{item}
                \item{enumerate}
                \item{item}
            \end{itemize}
        \end{framed}

        \Lx{} numbered lists are very easy. Simply enclose the list in th \texttt{enumerate} environment, and place each list item as an argument to the \texttt{item} command.

    \begin{enumerate}
        \item{Washington}
        \item{Adams}
        \item{Jefferson}
        \item{Madison}
    \end{enumerate}

        \begin{verbatim}
\documentclass{article}
    \title{Unnumbered List}
    \author{Charles Carter}
    \date{\today{}}
\begin{document} 
    \maketitle{}
    \begin{enumerate}
        \item{Washington}
        \item{Adams}
        \item{Jefferson}
        \item{Madison}
    \end{enumerate}
\end{document}    
        \end{verbatim}

        \paragraph{Exercise:} Make your own numbered list.

        \subsection{Dictionary Lists}
        \label{Dictionary Lists}
        
        \begin{framed}
            \begin{itemize}
                \index{description}
                \index{item}
                \item{description}
                \item{item}
            \end{itemize}
        \end{framed}

        \Lx{} dictionary lists are very easy. Simply enclose the list in th \texttt{description} environment, place the term in square brackets ([]), and place each list item as an argument to the \texttt{item} command.

    \begin{description}
        \item[C]{a procedural language}
        \item[Java]{an object oriented language}
        \item[Lisp]{a functional language}
        \item[JavaScript]{an event driven language}
        \item[Erlang]{a concurrent language}
    \end{description}

        \begin{verbatim}
\documentclass{article}
    \title{Unnumbered List}
    \author{Charles Carter}
    \date{\today{}}
\begin{document} 
    \maketitle{}
        \item[C]{a procedural language}
        \item[Java]{an object oriented language}
        \item[Lisp]{a functional language}
        \item[JavaScript]{an event driven language}
        \item[Erlang]{a concurrent language}
\end{document}    
        \end{verbatim}

        \paragraph{Exercise:} Make your own dictionary list.

        \subsection{Nested Lists}
        \label{Nested Lists}

        And of course, \Lx{} lists can be nested. Notice how ``smart'' the numbered lists are --- the number style depends on the heading level.

    \begin{enumerate}
        \item{17th century}
        \begin{enumerate}
            \item{Thirty Years War}
            \begin{description}
                \item[Gustavus Adolphus]{Sweden}
                \item[Wallenstein]{Hapsburg Austria}
                \item[Turenne]{France}
            \end{description}
        \item{War of the Grand Alliance}
            \begin{description}
                \item[William III]{Dutch Republic}
                \item[Eugene]{Hapsburg Austria}
                \item[Vauban]{France}
            \end{description}
        \end{enumerate}
        \item{18th century}
        \begin{enumerate}
            \item{War of the Spanish Succession}
            \begin{description}
                \item[Eugene]{Hapsburg Austria}
                \item[Malbourough]{England}
                \item[Villeroi]{France}
                \item[Maximilian II]{Bavaria}
            \end{description}
            \item{Seven Years War}
            \begin{description}
                \item[Frederick II]{Prussia}
                \item[Daun]{Hapsburg Austria}
                \item[Maximilian III]{Bavaria}
                \item[Clive]{Great Britain}
            \end{description}
        \end{enumerate}
        \item{19th century}
        \begin{enumerate}
            \item{War of the Sixth Coalition}
            \begin{description}
                \item[Napoleon I]{}
                \item[Blucher]{Prussia}
                \item[Bennigsen]{Russia}
                \item[Schwarenberg]{Austria}
            \end{description}
            \item{Franco-Prussian War}
            \begin{description}
                \item[Napoleon III]{}France
                \item[Moltke]{Prussia}
            \end{description}
        \end{enumerate}
        \item{20th century}
        \begin{enumerate}
            \item{World War 1}
            \begin{description}
                \item[Haig]{United Kingdom}
                \item[Foch]{France}
            \item[Hindenburg]{Germany}
                \item[Pershing]{United States}
            \end{description}
            \item{World War 2}
            \begin{description}
                \item[McArthur]{United States}
                \item[Montgomery]{United Kingdom}
                \item[Manstein]{Germany}
                \item[Rossokovsky]{Soviet Union}
            \end{description}
        \end{enumerate}
    \end{enumerate}

        \begin{verbatim}
\documentclass{article}
    \title{Unnumbered List}
    \author{Charles Carter}
    \date{\today{}}
\begin{document} 
    \maketitle{}
    \begin{enumerate}
        \item{17th century}
        \begin{enumerate}
            \item{Thirty Years War}
            \begin{description}
                \item[Gustavus Adolphus]{Sweden}
                \item[Wallenstein]{Hapsburg Austria}
                \item[Turenne]{France}
            \end{description}
        \item{War of the Grand Alliance}
            \begin{description}
                \item[William III]{Dutch Republic}
                \item[Eugene]{Hapsburg Austria}
                \item[Vauban]{France}
            \end{description}
        \end{enumerate}
        \item{18th century}
        \begin{enumerate}
            \item{War of the Spanish Succession}
            \begin{description}
                \item[Eugene]{Hapsburg Austria}
                \item[Malbourough]{England}
                \item[Villeroi]{France}
                \item[Maximilian II]{Bavaria}
            \end{description}
            \item{Seven Years War}
            \begin{description}
                \item[Frederick II]{Prussia}
                \item[Daun]{Hapsburg Austria}
                \item[Maximilian III]{Bavaria}
                \item[Clive]{Great Britain}
            \end{description}
        \end{enumerate}
        \item{19th century}
        \begin{enumerate}
            \item{War of the Sixth Coalition}
            \begin{description}
                \item[Napoleon I]{}
                \item[Blucher]{Prussia}
                \item[Bennigsen]{Russia}
                \item[Schwarenberg]{Austria}
            \end{description}
            \item{Franco-Prussian War}
            \begin{description}
                \item[Napoleon III]{}France
                \item[Moltke]{Prussia}
            \end{description}
        \end{enumerate}
        \item{20th century}
        \begin{enumerate}
            \item{World War 1}
            \begin{description}
                \item[Haig]{United Kingdom}
                \item[Foch]{France}
            \item[Hindenburg]{Germany}
                \item[Pershing]{United States}
            \end{description}
            \item{World War 2}
            \begin{description}
                \item[McArthur]{United States}
                \item[Montgomery]{United Kingdom}
                \item[Manstein]{Germany}
                \item[Rossokovsky]{Soviet Union}
            \end{description}
        \end{enumerate}
    \end{enumerate}
\end{document}    
        \end{verbatim}

        \paragraph{Exercise:} Make your own nested list to at least two levels.

        \subsection{Listings Package}
        \label{Listings Package}
        
%        \begin{framed}
%            \begin{itemize}
%                \item{}
%            \end{itemize}
%        \end{framed}
%
%
%        \begin{verbatim}
%\documentclass{article}
%    \title{This is My Title}
%    \author{Charles Carter}
%    \date{\today{}}
%\begin{document} 
%    \maketitle{}
%    \section{Introduction}
%    \label{Introduction}
%    \section{Body}
%    \label{Body}
%    \section{Conclusion}
%    \label{Conclusion}
%\end{document}    
%        \end{verbatim}
%
%        \paragraph{Exercise:}
%
%        \paragraph{Exercise:}

        \subsection{Listings Version 1}
        \label{Listings Version 1}
        
%        \begin{framed}
%            \begin{itemize}
%                \item{}
%            \end{itemize}
%        \end{framed}
%
%
%        \begin{verbatim}
%\documentclass{article}
%    \title{This is My Title}
%    \author{Charles Carter}
%    \date{\today{}}
%\begin{document} 
%    \maketitle{}
%    \section{Introduction}
%    \label{Introduction}
%    \section{Body}
%    \label{Body}
%    \section{Conclusion}
%    \label{Conclusion}
%\end{document}    
%        \end{verbatim}
%
%        \paragraph{Exercise:}
%
%        \paragraph{Exercise:}

        \subsection{Listings Version 2}
        \label{Listings Version 2}
        
%        \begin{framed}
%            \begin{itemize}
%                \item{}
%            \end{itemize}
%        \end{framed}
%
%
%        \begin{verbatim}
%\documentclass{article}
%    \title{This is My Title}
%    \author{Charles Carter}
%    \date{\today{}}
%\begin{document} 
%    \maketitle{}
%    \section{Introduction}
%    \label{Introduction}
%    \section{Body}
%    \label{Body}
%    \section{Conclusion}
%    \label{Conclusion}
%\end{document}    
%        \end{verbatim}
%
%        \paragraph{Exercise:}
%
%        \paragraph{Exercise:}
