%tables.tex
	
	\section{Tables}
	\label{Tables}

        Unfortunately, tables in \Lx{} tend to be a bit cumbersome. Tables are specified row by row, and require meticulous attention to detail. If sufficient care is taken, composing tables is not very difficult. Fortunately, simple tables are not too hard. 
        
        \subsection{Verbatim table}
        \label{Verbatim table}
        
        \begin{framed}
            \begin{itemize}
                \index{verbatim}
                \index{verbatim environment}
                \item{verbatim}
                \item{verbatim environment}
            \end{itemize}
        \end{framed}

        As with lists, ths simplest, most brain-dead way to make a table is by using the \texttt{verbatim} environment. The hardest part is making sure that your cells align properly using correct spacing. Hint: use a monospace font in your text editor rather than a variable spaced font.

        \begin{verbatim}
Abigail     Derek       Gladiola
Brenda      Edgar       Hibiscus
Claudia     Frank       Impatiens
        \end{verbatim}

        \begin{realverbatim}
\documentclass{article}
    \title{Verbatim Tables}
    \author{Charles Carter}
    \date{\today{}}
\begin{document} 
    \maketitle{}
        \begin{verbatim}
Abigail     Derek       Gladiola
Brenda      Edgar       Hibiscus
Claudia     Frank       Impatiens
        \end{verbatim}
\end{document}    
        \end{realverbatim}

        \paragraph{Exercise:} Create your own verbatim table.

        \subsection{Simple table}
        \label{Simple table}
        
        \begin{framed}
            \begin{itemize}
                \index{tabular environment}
                \index{l (column specifier)}
                \index{c (column specifier)}
                \index{r (column specifier)}
                \index{\textbackslash\textbackslash (row terminator)}
                \index{\& (cell separator)}
                \item{tabular environment}
                \item{l (column specifier)}
                \item{c (column specifier)}
                \item{r (column specifier)}
                \item{\textbackslash\textbackslash{} (row terminator)}
                \item{\& (cell separator)}
            \end{itemize}
        \end{framed}

        A simple \Lx{} table is created within a \texttt{tabular} environment. The column specifications are passed as a parameter to te environment --- r (right), c (center), l (left). Table rows are terminated by two backslashes (\textbackslash\textbackslash). Cells are separated by an ampersand (\&).

        \begin{tabular}{l c r}
Ann &  Derek &   Gladiola \\
Brenda &  Ed &   Hibiscus \\
Claudette &  Frederick &   Impatiens \\
        \end{tabular}


        \begin{verbatim}
\documentclass{article}
    \title{Smple Tables}
    \author{Charles Carter}
    \date{\today{}}
\begin{document} 
    \maketitle{}
        \begin{tabular}{l c r}
Ann         & Derek         & Gladiola \\
Brenda      & Ed            & Hibiscus \\
Claudette   & Frederick     & Impatiens \\
        \end{tabular}
\end{document}    
        \end{verbatim}

        \paragraph{Exercise:} Make your own simple table.

        \subsection{Row and column lines}
        \label{Row and column lines}
        
        \begin{framed}
            \begin{itemize}
                \index{\textbar}
                \index{hline}
                \item{\textbar}
                \item{hline}
            \end{itemize}
        \end{framed}

        Table rules are specified by the vertical bar (\textbar) and by \texttt{hline}. Column rules are set in the parameter of the column specifications passed to the tabular environment. Row rules are created by the \texttt{hline} command. Both of these can be doubled to produce double rules.
        

    \begin{tabular}{| l | r || l | r |}
        \hline
        Girls & Boys & Flowers & Animals\\
        \hline
        \hline
        Abigail &  Derek &   Gladiola & Jackel \\
        \hline
        Brenda &  Edgar &   Hibiscus & Koala \\
        \hline
        Claudia &  Frank &   Impatiens & Lynx \\
        \hline
    \end{tabular}


         \begin{verbatim}
\documentclass{article}
    \title{Table Rules}
    \author{Charles Carter}
    \date{\today{}}
\begin{document} 
    \maketitle{}
    \begin{tabular}{| l | r || l | r |}
        \hline
        Girls & Boys & Flowers & Animals\\
        \hline
        \hline
        Abigail &  Derek &   Gladiola & Jackel \\
        \hline
        Brenda &  Edgar &   Hibiscus & Koala \\
        \hline
        Claudia &  Frank &   Impatiens & Lynx \\
        \hline
    \end{tabular}
\end{document}    
        \end{verbatim}        

        \paragraph{Exercise:} Create your own simple table with vertical ane horizontal rules.

        \subsection{Column spacing}
        \label{Column spacing}
        
        \begin{framed}
            \begin{itemize}
                \index{p (column specifier)}
                \item{p (column specifier)}
            \end{itemize}
        \end{framed}

        The column specifiers \texttt{l} (left), \texttt{c} (center), and \texttt{}r (right) mostly do what you want, but their capacity is limited and frequently you need to use the column specifier \texttt{p} (paragraph) to make the content of the cells behave. \texttt{p} takes as a parameter a length recognized by \Lx{}, frequently a common English measure expressed in inches, for example, \texttt{p\{2in\}}.

        \begin{table}[!ht]
            \label{P columns}
        \begin{tabular}{ p{1in} p{4in}}
            \hline{}
            C & C is a general-purpose, imperative computer programming language, supporting structured programming, lexical variable scope and recursion, while a static type system prevents many unintended operations. \\
            Java & Java is a general-purpose computer programming language that is concurrent, class-based, object-oriented, and specifically designed to have as few implementation dependencies as possible. It is intended to let application developers ``write once, run anywhere'' meaning that compiled Java code can run on all platforms that support Java without the need for recompilation. \\
            Common Lisp & Common Lisp (historically, LISP) is a family of computer programming languages with a long history and a distinctive, fully parenthesized prefix notation. Originally specified in 1958, Lisp is the second-oldest high-level programming language in widespread use today. Only Fortran is older, by one year. \\
            JavaScript & JavaScript is a high-level, dynamic, untyped, and interpreted programming language. It has been standardized in the ECMAScript language specification. \\
            Erlang & Erlang is a programming language designed for developing robust systems of programs that can be distributed among different computers in a network. Named for the Danish mathematician Agner Krarup Erlang, the language was developed by the Ericsson Computer Sciences Lab to build software for its own telecommunication products.  \\
            \hline{}
        \end{tabular}
            \caption{P columns}
        \end{table}

        \begin{verbatim}
\documentclass{article}
    \title{This is My Title}
    \author{Charles Carter}
    \date{\today{}}
\begin{document} 
    \maketitle{}
        \begin{tabular}{ p{1in} p{4in}}
            \hline{}
            C & C is a general-purpose, imperative computer programming 
            language, supporting structured programming, lexical variable 
            scope and recursion, while a static type system prevents many 
            unintended operations. \\
            Java & Java is a general-purpose computer programming language 
            that is concurrent, class-based, object-oriented, and 
            specifically designed to have as few implementation 
            dependencies as possible. It is intended to let application 
            developers "write once, run anywhere" meaning that compiled 
            Java code can run on all platforms that support Java without 
            the need for recompilation. \\
            Common Lisp & Common Lisp (historically, LISP) is a family 
            of computer programming languages with a long history and a 
            distinctive, fully parenthesized prefix notation. Originally 
            specified in 1958, Lisp is the second-oldest high-level 
            programming language in widespread use today. Only Fortran 
            is older, by one year. \\
            JavaScript & JavaScript is a high-level, dynamic, untyped, 
            and interpreted programming language. It has been 
            standardized in the ECMAScript language specification. \\
            Erlang & Erlang is a programming language designed for 
            developing robust systems of programs that can be 
            distributed among different computers in a network. 
            Named for the Danish mathematician Agner Krarup Erlang, 
            the language was developed by the Ericsson Computer 
            Sciences Lab to build software for its own tele
            communication products.  \\
            \hline{}
        \end{tabular}
\end{document}    
        \end{verbatim}

        \paragraph{Exercise:} Create your own table with paragraph spacing.

        \subsection{Captions and labels}
        \label{Captions and labels}
        
        \begin{framed}
            \begin{itemize}
                \index{table environment}
                \index{caption}
                \index{label}
                \index{listoftables}
                \item{table environment}
                \item{caption}
                \item{label}
                \item{listoftables}
            \end{itemize}
        \end{framed}

        The \textit{table environment} is created with the \texttt{begin\{table\}} command, and ended with the \texttt{end\{table\}} command. Table environments contain tables (inside the \textit{tabular environment}), and optional labels and captions. A caption is created with the \texttt{caption} command. Notice that in the example below the label command is placed \textit{after} the caption command. Table \ref{Names} is on \pageref{Names} and table \ref{Plants} is on page \pageref{Plants}.

        The \texttt{listoftables} command, which should be placed just after the \texttt{tableofcontants} command, generates a list of the tables in the document. For an example, see the table of contents of this tutorial and the listing below.

    \begin{table}
    \caption{Names - Caption Above}
    \label{Names}
        \begin{tabular}{r | l l}
            \hline
            A & Akin & Allie \\
            B & Bobby & Bonita \\
            C & Chad & Carly \\
            \hline
        \end{tabular}
    \end{table}

    \begin{table}
        \begin{tabular}{r | l l}
            \hline
            D & Date & Dogwood  \\
            E & Eggplant & Elm \\
            F & Fig & Frasier Fir \\
            \hline
        \end{tabular}
    \caption{Plants - Caption Below}
    \label{Plants}
    \end{table}

        \begin{verbatim}
\documentclass{article}
    \title{Table Captions and Labels}
    \author{Charles Carter}
    \date{\today{}}
\begin{document} 
    \maketitle{}
    \listoftables{}
Reference to table \ref{Plants} on page \pageref{Plants}.
    \begin{table}
        \caption{Names - Caption Above}
        \label{Names}
        \begin{tabular}{r | l l}
            \hline
            A & Akin & Allie \\
            B & Bobby & Bonita \\
            C & Chad & Carly \\
            \hline
        \end{tabular}
    \end{table}
    \newpage
    \begin{table}
        \begin{tabular}{r | l l}
            \hline
            D & Date & Dogwood  \\
            E & Eggplant & Elm \\
            F & Fig & Frasier Fir \\
            \hline
        \end{tabular}
        \caption{Plants - Caption Below }
        \label{Plants}
    \end{table}
    Reference to table \ref{Names} on page \pageref{Names}.
\end{document}    
        \end{verbatim}

        \paragraph{Exercise:} You can refer to a table, and the page number of the table, by using the \texttt{ref} and \texttt{pageref} commands. See section \ref{Text references} on page \pageref{Text references}.

        \paragraph{Exercise:} How would you center a table horizontally on the page?



        \subsection{Multicells}
        \label{Multicells}
        
        \begin{framed}
            \begin{itemize}
                \index{multicolumn}
                \index{multirow package}
                \index{multirow}
                \item{multicolumn}
                \item{multirow package}
                \item{multirow}
            \end{itemize}
        \end{framed}

        \Lx{} tables can be nested, but for multi-column and multi-row spans, there are separate commands that are much easier to use. Multicolumn cells are created by the \texttt{multicolumn\{num-of-cols\}\{col-alignment\}\{cell-content\}} as illustrated in table \ref{multicol} on page \pageref{multicol}.

        \begin{table}
            \begin{tabular}{| l | p{2.5in} |  c |}
                \hline
                \multicolumn{3}{| c |}{Measures of center} \\
                \hline
                mean & the sum of all values divided by the number of values & $\frac{\sum_{i=1}^{i=n} x_i}{n}$ \\
                \hline
                median & the middle value in a vector of sorted value & $\frac{length(V) + 1}{n} | sorted(V)$  \\
                \hline
                mode & the most frequent value or values if more than one  & (no formula) \\
                \hline
                \multicolumn{3}{| c |}{Measures of dispersion} \\
                \hline
                variance & the sum of the difference between each value of the vector and the mean of the vector divided by the number of values in the vector & $\sum_{i=1}^{i=n} \frac{V_i - \mu}{n}$ \\
                \hline
                standard deviation & the square root of the sum of the difference between each value of the vector and the mean of the vector divided by the number of values in the vector  & $\sqrt{\sum_{i=1}^{i=n} \frac{V_i - \mu}{n}}$ \\
                \hline
                median absolute deviation & the average distance between each data value and the mean & $\frac{\sum_{i=1}^{i=n}|V_i - \mu|}{n}$  \\ 
                \hline
            \end{tabular}
        \caption{Multicolumn Example}
        \label{multicol}
        \end{table}

        \begin{verbatim}
\documentclass{article}
    \title{This is My Title}
    \author{Charles Carter}
    \date{\today{}}
\begin{document} 
    \begin{table}
        \begin{tabular}{| l | p{2.5in} |  c |}
            \hline
            \multicolumn{3}{| c |}{Measures of center} \\
            \hline
            mean & 
                the sum of all values divided by the number of values & 
                $\frac{\sum_{i=1}^{i=n} x_i}{n}$ \\
            \hline
            median & 
                the middle value in a vector of sorted value & 
                $\frac{length(V) + 1}{n} | sorted(V)$  \\
            \hline
            mode & 
                the most frequent value or values if more than one  
                & (no formula)  \\
            \hline
            \multicolumn{3}{| c |}{Measures of dispersion} \\
            \hline
            variance & 
                the sum of the difference between each value of the vector 
                and the mean of the vector divided by the number of values 
                in the vector & 
                $\sum_{i=1}^{i=n} \frac{V_i - \mu}{n}$ \\
            \hline
            standard deviation & 
            the square root of the sum of the difference between each value 
            of the vector and the mean of the vector divided by the number 
            of values in the vector  & 
            $\sqrt{\sum_{i=1}^{i=n} \frac{V_i - \mu}{n}}$ \\
            \hline
            median absolute deviation & the average distance between each data value and the mean & $\frac{\sum_{i=1}^{i=n}|V_i - \mu|}{n}$  \\ 
            \hline
        \end{tabular}
    \caption{Multicolumn Example}
    \label{multicol}
    \end{table}
\end{document}    
        \end{verbatim}

        \paragraph{Exercise:}

        \paragraph{Exercise:}

        \subsection{Long tables}
        \label{Long tables}
        
        \begin{framed}
            \begin{itemize}
                \item{}
            \end{itemize}
        \end{framed}


        \begin{verbatim}
\documentclass{article}
    \title{This is My Title}
    \author{Charles Carter}
    \date{\today{}}
\begin{document} 
    \maketitle{}
    \section{Introduction}
    \label{Introduction}
    \section{Body}
    \label{Body}
    \section{Conclusion}
    \label{Conclusion}
\end{document}    
        \end{verbatim}

        \paragraph{Exercise:}

        \paragraph{Exercise:}

