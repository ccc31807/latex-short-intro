\documentclass{article}
    \usepackage{amsmath}
    \newtheorem{theorem}{Theorem}
    \newtheorem{proof}{Proof}
    \newtheorem{definition}{Definition}
    \title{Theorems, Proofs, Definitions, etc.}
    \author{Charles Carter}
    \date{\today{}}
\begin{document}
    \maketitle{}
    Theorems, proofs, definitions, etc.  can easily be defined
    \begin{theorem}
        \label{thm:first}
        \addcontentsline{toc}{subsubsection}{Fundamental Theorem of Calculus}
        Definite integral of a function is related to its antiderivative, 
        and can be reversed by differentiation.
    \end{theorem}
    \begin{proof}
        \label{prf:first}
        \addcontentsline{toc}{subsubsection}{Proof of Fundamental Theorem of Calculus}
        If $f$ is continuous on $[a, b]$, 
            then $\int_a^b f$ exists. \\
        If $f$ is continous on $[a, b]$ and $c \in [a, b]$, 
            then $\int_a^c F + \int_c^b F = \int_a^b F$. \\
        If $m \leq f \leq M$ on $[a, b]$, 
            then $(b - a)m \leq \int_a^b f \leq (b - a)M$.
    \end{proof}
    \begin{definition}
        \label{def:first}
        \addcontentsline{toc}{subsubsection}{Definition of Calculus}
        Calculus is the branch of mathematics that deals with the 
        finding and properties of derivatives and integrals of 
        functions, by methods originally based on the summation 
        of infinitesimal differences. The two main types are 
        differential calculus and integral calculus.
    \end{definition}
    For the definition of calculus, see definition \ref{def:first}. 
    For the fundamental theorem of calculus, see theorem \ref{thm:first}. 
    For the proof, see proof \ref{prf:first}.
    \end{document}

